\chapter{Sheaves in homotopy type theory}
\label{chap:sheaf}

\section{Forcing in type theory}
\label{sec:forcing}

\section{Sheaves in topoi}
\label{sec:sheaf_topos}

In this section, we will rather work in an arbitrary topos rather in type theory. The next section will present a
generalisation of the results presented here.

Let us fix for the whole section a topos $\mathcal E$, with subobject
classifier $\Omega$. A {\em Lawvere-Tierney topology} on $\mathcal E$
is a way to modify slightly truth values of $\mathcal E$. It allows to
speak about {\em locally true} things instead of {\em true} things.

\begin{defi}[Lawvere-Tierney topology~\cite{maclanemoerdijk}]\label{defi:LT}
  A Lawvere-Tierney topology is an endomorphism $j:\Omega \to \Omega$
  preserving $\True$ ($j \ \True = \True$), idempotent ($j\circ j =
  j$) and commuting with products ($j \circ \wedge = \wedge \circ (j,j)$).
\end{defi}

A classical example of Lawvere-Tierney topology is given by double
negation. Other examples are given by Grothedieck topologies, in the
sense
\begin{thm}[{\cite[V.1.2]{maclanemoerdijk}}]
  Every Grothendieck topology $J$ on a small category $\mathbf C$ determines a
  Lawvere-Tierney topology $j$ on the presheaf topos
  $\mathbf{Sets}^{\mathbf C^{\mathbf{op}}}$.
\end{thm}

Any Lawvere-Tierney topology $j$ on $\mathcal E$ induces a closure operator
$A \mapsto \closure{A}$ on subobjects. If we see a subobject $A$ of $E$
as a characteristic function $\Char{A}$, the closure $\closure{A}$
corresponds to the subobject of $E$ whose characteristic function is 
%
\[
\Char{\closure{A}} = j \circ \Char{A}.
\]%
%
A subobject $A$ of $E$ is said to
be dense when $\closure{A} = E$.

Then, we are interested in objects of $\mathcal E$ for which it is
impossible to make a distinction between objects and their dense
subobjects, \ie{} for which ``true'' and ``locally true''
coincide. Such objects are called {\em sheaves}, and are defined as

\begin{defi}[Sheaves{\cite[V.2]{maclanemoerdijk}}]
  On object $F$ of $\mathcal E$ is a sheaf (or $j$-sheaf) if for every
  dense monomorphism $m: A \hookrightarrow E$ in $\mathcal E$, the
  canonical map $\Hom{\mathcal E}(E,F) \rightarrow \Hom{\mathcal E}(A,F)$ is an
isomorphism.
\end{defi}

One can show that $\Sh{\mathcal E}$, the full sub-category of
$\mathcal E$ given by
sheaves, is again a topos, with classifying object
%
\[
\Omega_j = \{ P \in \Omega \ | \ j P  = P \}.
\]

Lawvere-Tienrey sheafification is a way to build a left adjoint $\mathbf{a}_j$ to the
inclusion $\mathcal E \hookrightarrow \Sh(\mathcal E)$, exhibiting
$\Sh(\mathcal E)$ as a reflective subcategory of $\mathcal E$. In
particular, that implies that logical principles valid in $\mathcal E$
are still valid in $\Sh(\mathcal E)$.

For any object $E$ of $\mathcal E$, $\mathbf{a}_j(E)$ is defined as in
the following diagram
\[
  \xymatrix{ 
    E \ar[rr]^{\{\cdot\}_E} \ar@{->>}[d]_{\theta_E} && \Omega^E \ar[d]^{j^E}\\
    E' \ar@{^{(}->}[rr] \ar[dr]_{\text{closure}} && \Omega_j^E \\
    & \mathbf{a}_j(E) \ar[ur]&
  }
\]

The proof that $\mathbf a_j$ defines a left adjoint to the inclusion
can be found in~\cite{maclanemoerdijk}.


\section{Sheaves in homotopy type theory}
\label{sec:sheaf_hott}