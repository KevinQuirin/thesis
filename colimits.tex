\chapter{Colimits}
\label{chap:colim}

\epigraph{A comathematician is a device turning cotheorems into
  ffee.}{Mathematical folklore}

As seen in chapter~\ref{chap:hott}, adding sigma-types to type theory
results in adding limits over graphs in the underlying category, and
adding higher inductives typves results in adding colimits over
graphs. If limits has been extensively studied in~\cite{lumsdaine},
theory of colimits was not completely treated.

The following is conjoint work with Simon Boulier and Nicolas
Tabareau, helped by precious discussions with Egbert Rijke.
The sections~\ref{sec:colim} and~\ref{sec:floris} are extended version
of the blog post~\cite{boulier}.

\section{Colimits over graphs}
\label{sec:colim}

As colimits are just dual to limits, it seems that it would be very
easy to translate the work on limits to colimits. Althought, even if
it might be because we are more habituated to manipulate sigma-types
than higher inductive types, it seems way harder.

\subsection{Definitions}
\label{ssec:colim:defi}

Let's recall the definitions of graphs and diagrams over graphs,
introduced in~\cite{lumsdaine}.

\begin{defi}[Graph]\label{defi:graph}
  A {\em graph} $G$ is the data of
  \begin{itemize}
  \item a type $G_0$ of vertices ;
  \item for any $i,j:G_0$, a type $G_1(i,j)$ of edges.
  \end{itemize}
\end{defi}

\begin{defi}[Diagram]\label{defi:diagram}
  A {\em diagram} $D$ over a graph $G$ is the data of
  \begin{itemize}
  \item for any $i:G_0$, a type $D_0(i)$ ;
  \item for any $i,j:G_0$ and all $\phi : G_1(i,j)$, a map $D_1(\phi)
    : D_0(i) \to D_0(j)$
  \end{itemize}
\end{defi}

When the context is clear, $G_0$ will be simply denoted $G$,
$G_1(i,j)$ will be noted $G(i,j)$, $D_0(i)$
will be noted $D(i)$ or $D_i$, and $D_1(\phi)$ will be noted
$D_{i,j}(\phi)$ or simply $D(\phi)$ ($i$ and $j$ can be inferred from
$\phi$).

\begin{exms}
  \item One can consider the following graph, namely the graph of
    (co)equalizers
    \[ \xymatrix{\bullet \ar@<-.5ex>[r] \ar@<.5ex>[r] & \bullet} \]
    Here, $G_0 = \two$, $G_1(\True,\False) = \two$ and other
    $G_1(i,j)$ are empty.

    A diagram over this graph consists of two types $A$ and $B$, and
    two maps $f,g:A \to B$, producing the diagram
    \[ \xymatrix{A \ar@<-.5ex>[r]_g \ar@<.5ex>[r]^f & B} \]
  \item The graph of the mapping telescope is 
    \[ \xymatrix{ \bullet \ar[r] &\bullet \ar[r] &\cdots} \]
    In other words, $G_0=\N$ and $G_1(i,i+1) = \one$.

    A diagram over the mapping telescope is a sequence of types $P:\N
    \to \Type$ together with arrows $f_n:P_n \to P_{n+1}$:
    \[ \xymatrix{ P_0 \ar[r]^{f_0} & P_1 \ar[r]^{f_1} & \cdots} \]
\end{exms}

What we would like now would be to define the colimits of these
diagrams over graphs, that would satisfy type theoretic versions of
usual properties: it should make the diagram commute, and be universal
with respect to this property.
From now on, let $G$ be a graph and $D$ a diagram over this graph.

The commutation of the diagram is easy: the colimit should be the tip
of a cocone.

\begin{defi}[Cocone]\label{defi:cocone}
  Let $Q$ be a type. A cocone over $D$ intro $Q$ is the data of arrows
  $q_i:D_i \to Q$, and for any $i,j:G$ and $g:G(i,j)$, an homotopy 
  $q_j \circ D(g) \homot q_i$.
\end{defi}

If $Q$ and $Q'$ are type with an arrow $f:Q\to Q'$, and if $C$ is a
cocone over $D$ into $Q$, one can easily build a cocone on $D$ into
$Q'$ by postcomposing all maps of the cocone by $f$, giving a map
\newcommand{\postcomposecocone}{\mathrm{postcompose}_{\mathrm{cocone}}}
\[\postcomposecocone : \cocone_D(Q) \to (Q':\Type) \to (Q \to Q') \to \cocone_D(Q') \]
The other way around (from a cocone into $Q'$, give a map $Q\to Q'$)
is exactly the second condition we seek:

\begin{defi}[Universality of a cocone]
  Let $Q$ be a type, and $C$ be a cocone over $D$ into $Q$. $C$ is
  said universal if for any type $Q'$, $\postcomposecocone(C,Q')$ is
  an equivalence.
\end{defi}

We can finally define what it means for $Q$ to be a colimit of $D$.

\begin{defi}[Colimit]\label{defi:colimit}
  A type $Q$ is said to be a colimit of $D$ if there is a cocone $C$
  over $D$ into $Q$, which is universal.
\end{defi}

\begin{exm}
  Let $A$, $B$ be types and $f,g:A\to B$.
  Let $Q$ be the HIT generated by
  $\left|
  \begin{array}{lll}
    q & : & B \to Q \\
    \alpha & : & q \circ f \homot q \circ g
  \end{array} \right. .$
Then $Q$ is a colimit of the coequalizer diagram associated to
$A,B,f,g$. We say that $Q$ is a coequalizer of $f$ and $g$.
\end{exm}

Note that for any diagram $D$, one can build a free colimit of $D$,
namely the higher inductive type $\colim(D)$ generated by
\[ 
  \left|
    \begin{array}{lll}
      \colim & : & \prodD i G {D_i \to \colim(D)} \\
      \alpha_\colim & : & \prodD {i j} G {\prodD g {G(i,j)} {\prodD x
                          {D_i} {\colim_j \circ D(g) \homot \colim_i}}}
    \end{array} \right.
\]
\todo[inline]{Finish colimits}
\subsection{Properties of colimits}
\label{ssec:prop_colim}

\subsection{Towards highly coherent colimits}
\label{ssec:high_colimit}


\section{Van Doorn's and Boulier's constructions}
\label{sec:floris}

\begin{prop}\label{prop:cech}
  
\end{prop}

\section{Towards groupoid objects}
\label{sec:groupoid}