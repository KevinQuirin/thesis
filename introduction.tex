\chapter{Introduction}
\label{chap:intro}

\todo[inline]{Write the intro}

Things to say:

One goal of this thesis is to define truncated version of concepts of
homotopy type theory, such as colimits and modalities. Then, using
them, we define an extension of the not-not Gödel translation, {\em
  via} the theory of Lawvere-Tierney sheaves.

To be easily readable, some references are called by their usual
names: \cite{hottbook} is the Homotopy Type Theory book,
\cite{hottlib} is the Coq/HoTT library, \cite{lurie} is Lurie's
monograph Higher topos theory, \etc 

In chapter~\ref{chap:hott}, we recall the basic definitions in
homotopy type theory, introducing notations to be consistent in the
whole thesis. Chapter~\ref{chap:modalities} introduces theory of
modalities and truncated modalities, as well as the translation of
type theories induced by modalities. Chapter~\ref{chap:colim} defines
a basic theory of colimits in homotopy type
theory. Chapter~\ref{chap:sheaves} builds the sheafification functor,
extending Gödel translation to all truncated types. Finally,
chapter~\ref{chap:conclusion} concludes and presents the future works.