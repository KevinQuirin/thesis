\chapter{Conclusion and future works}
\label{chap:conclusion}

\section{Conclusion}
\todo[inline]{Write a conclusion.}
\section{Future works}

\paragraph*{Extension to Type.}
At the moment, our sheafification functor only handles truncated type,
and we have to compose it with truncations. It would be way more
satisfying to be able to define it on whole $\Type$ left-exactly. The
main issue is that some types, which are not $n$-truncated for any
$n$, are not even the limit of their
truncations~\cite{morelvv}. Therefore, there seems to be no way to
create a link between a non-truncated type and truncated types, to
extend our inductive definition. 

\paragraph*{Lawvere-Tierney sheaves in higher topos theory.}
If we rely on the leitmotiv
\begin{quote}
  ``Homotopy type theory is the internal language of $(\infty,1)$-topos,''
\end{quote}
we could transpose our work to higher topos theory. As there are more
tools in topos theory (\eg{} we can access the definitional equality),
it could be a first step in solving the previous future work.

\paragraph*{Lawvere-Tierney subsumes Grothendieck?}
In topos theory, there are two different notions of sheaves: the
Grothendieck sheaves and the Lawvere-Tierney sheaves.
The former is a topological, geometrical concept, while the latter is
rather a logical concept. 
We have the following:
\begin{thm}[{\cite[{Section V.4, theorem 2}]{maclanemoerdijk}}]
  Let $\mathbf{C}$ is a small category and $j$ a Lawvere-Tierney
  topology on $\mathbf{Sets}^{\mathbf{C}^{\mathrm{op}}}$, while $J$ is
  the corresponding Grothendieck topology on $\mathbf{C}$. Then a
  presheaf $P$ is a sheaf for $j$ iff $P$ is a $J$-sheaf.
\end{thm}





%%% Local Variables:
%%% mode: latex
%%% TeX-master: "main"
%%% End:
