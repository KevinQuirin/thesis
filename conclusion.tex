\chapter{Conclusion and future works}
\label{chap:conclusion}

\section{Conclusion}
\todo[inline]{Write a conclusion.}
\section{Future work}

\paragraph*{Extension to Type.}
At the moment, our sheafification functor only handles truncated type,
and we have to compose it with truncations. It would be way more
satisfying to be able to define it on whole $\Type$ left-exactly. The
main issue is that some types, which are not $n$-truncated for any
$n$, are not even the limit of their
truncations~\cite{morelvv}. Therefore, there seems to be no way to
create a link between a non-truncated type and truncated types, to
extend our inductive definition.
It might be possible to have such a link using axioms such as
Whitehead's principle~\cite[Section 8.8]{hottbook} or Postnikov
principle~\cite[Section 5.5.6]{lurie}, and use it to build a real
modality on $\Type$.

\paragraph*{Lawvere-Tierney sheaves in higher topos theory.}
If we rely on the leitmotiv
\begin{quote}
  Homotopy type theory is the internal language of $(\infty,1)$-topos,
\end{quote}
we could transpose our work to higher topos theory. As there are more
tools in topos theory (\eg{} we can access the definitional equality),
it could be a first step in solving the previous future work.
This kind of ``reverse engineered'' proof has already been done for a
proof of the Blakers-Massey theorem by Charles Rezk~\cite{rezk-BM},
inspired by the homotopy-type-theoretic proof by Peter LeFanu
Lumsdaine, Eric Finster and Dan Licata.

\paragraph*{Lawvere-Tierney subsumes Grothendieck?}
In topos theory, there are two different notions of sheaves: the
Grothendieck sheaves and the Lawvere-Tierney sheaves.
The former is a topological, geometrical concept, while the latter is
rather a logical concept.
Grothendieck sheaves are based on {\em Grothendieck
  topologies}~\cite[Chapter III]{maclanemoerdijk}, and one can show
that Lawvere-Tierney topologies on a presheaf topos
$\mathbf{Sets}^{\mathbf{C}^{\mathrm{op}}}$ correspond exactly to
Grothendieck topologies on $\mathbf{C}$. Then, we have the following:
\begin{thm}[{\cite[{Section V.4, theorem 2}]{maclanemoerdijk}}]
  \label{thm:subsume}
  Let $\mathbf{C}$ is a small category and $j$ a Lawvere-Tierney
  topology on $\mathbf{Sets}^{\mathbf{C}^{\mathrm{op}}}$, while $J$ is
  the corresponding Grothendieck topology on $\mathbf{C}$. Then a
  presheaf $P$ is a sheaf for $j$ iff $P$ is a $J$-sheaf.
\end{thm}
The concept of Grothendieck sheaf and Grothendieck sheafification
already exists in $(\infty,1)$-topos~\cite[Section 6.2.2]{lurie}. 
It would be nice to check if theorem~\ref{thm:subsume} still holds,
either in the setting of homotopy type theory or in
the setting of higher topos. The former requires to formalize
Grothendieck topologies, sheaves and sheafification from higher topos
theory to homotopy type theory, while the latter requires to work on
the previous point.





%%% Local Variables:
%%% mode: latex
%%% TeX-master: "main"
%%% End:
