\chapter{Homotopy type theory}
\label{chap:hott}

\section{Dependent type theory}
\label{sec:mltt}
$a:A$: judgement ``$a$ is of type
  $A$''\nomenclature{$a:A$}{Judgement ``$a$ is of type $A$''}

\subsection{Empty and Unit types}
\label{ssec:unit_empty}

The first two types we will see are the Empty type (denoted
$\zero$\nomenclature{$\zero$}{Empty type}) and the Unit type (denoted
$\one$\nomenclature{$\one$}{Unit type}).
These are respectively the types with zero and one elements (named $\unittt$). Those two
types are dual to each other:
\begin{itemize}
\item having a term of type $\zero$ in the context allows to prove
  anything, while having a term of type $\one$ in the context is
  useless
\item dually, giving a term of type $\one$ is trivial, while giving a
  term of type $\zero$ is impossible (if the theory if consistent).
\end{itemize}

Here are the (non-trivial) introduction and elimition rules for these types:

\begin{center}
  \AxiomC{$\Gamma\vdash x:\zero$}
  \AxiomC{$X:\Type$}
  \RightLabel{$\zero$-\textsc{elim}}
  \BinaryInfC{$\Gamma \vdash X$}
  \DisplayProof
  \qquad
  \AxiomC{}
  \RightLabel{$\one$-\textsc{intro}}
  \UnaryInfC{$\Gamma\vdash \unittt:\one$}
  \DisplayProof
\end{center}

Under the propositions-as-types principle, $\zero$ is the type always
false, and $\one$ the type always true. With a categorical point of
view, $\zero$ is an initial object and $\one$ is a terminal object.

\subsection{Coproduct}
\label{ssec:coproduct}

The coproduct of $A$ and $B$, noted $A+B$\nomenclature{$A+B$}{Coproduct of
  types}, is seen as the disjoint sum of $A$ and $B$. It is described
by the inductive type generated by
\[ \left|
    \begin{array}{lll}
      \inl & : & A \to A+B \\
      \inr & : & B \to A+B
    \end{array}
  \right. \]

The introduction and elimination rules of coproduct are:
\begin{center}
  \AxiomC{$\Gamma\vdash a:A$}
  \RightLabel{$+$-$\textsc{intro}_L$}
  \UnaryInfC{$\Gamma\vdash \inl a : A+B$}
  \DisplayProof
  \qquad
  \AxiomC{$\Gamma\vdash b:B$}
  \RightLabel{$+$-$\textsc{intro}_R$}
  \UnaryInfC{$\Gamma\vdash \inr b : A+B$}
  \DisplayProof
  \vspace{1em}

  \AxiomC{$\Gamma\vdash p:A+B$}
  \AxiomC{$\Gamma,x:A\vdash c_A:C$}
  \AxiomC{$\Gamma,x:B\vdash c_B:C$}
  \RightLabel{$+$-\textsc{elim}}
  \TrinaryInfC{$\Gamma,\vdash\mathrm{sum\_rect}(p,c_A,c_B) : C$}
  \DisplayProof
\end{center}

Under the propositions-as-types principle, $A+B$ is seen as the
disjunction of $A$ and $B$.

\subsection{Dependent product}
\label{ssec:pi}
One of the things both mathematicians and computer scientists love to
do is to define functions. If $A$ and $B$ are types, one can consider
the type of functions from $A$ to $B$, taking an inhabitant of type
$A$ (called the source type) and giving an inhabitant of type $B$ (the
target type). What is new in dependent type theory
is that the target type is allowed to depend on the argument of the
function. 

\begin{exm}
  For example, one can consider a function taking a natural number $n$
  and giving a natural number greater than $n$. The target source
  depends indeed of $n$.
\end{exm}

The type of dependent functions with source type $A$ and target type
$B\, x$ is called ``dependent product over $B$'' (or ``pi-type over
$B$''), and will be denoted $\prodD x A {B\, x}$\nomenclature{$\prodD a A
  {B\, a}$}{Dependent product over $B$} or $(x:A)\to (B\,
a)$\nomenclature{$(x:A)\to (B\, a)$}{Dependent product over $B$}.
When $B$ does not depend on $A$, we just say ``arrow type'' and note
it $A\to B$\nomenclature{$A\to B$}{Type of arrows from $A$ to $B$}.

The introduction and elimination rules of dependent products are:

\begin{center}
  \AxiomC{$\Gamma,x:A\vdash b:B$}
  \RightLabel{$\prod$-\textsc{intro}}
  \UnaryInfC{$\Gamma\vdash \lambda\, (x:A),\, b : \prodD x A B$}
  \DisplayProof
  \qquad
  \AxiomC{$\Gamma\vdash f:\prodD x A B$}
  \AxiomC{$\Gamma\vdash a:A$}
  \RightLabel{$\prod$-\textsc{elim}}
  \BinaryInfC{$\Gamma\vdash f(a):B[a/x]$}
  \DisplayProof
\end{center}

Under the propositions-as-types, type of non-dependent functions $A\to
B$ is seen as implication $A\To B$, and type of dependent functions
$\prodD x A {B\, x}$ is seen as universally quantified formulas
$\forall x,\, B\, x$.

\subsection{Dependent sum}
\label{ssec:sigma}

If $A$ and $B$ are two types, we would like to define the type of
pairs $(a,b)$, where $a:A$ and $b:B$. The resulting type is called the
product of $A$ and $B$, noted $A\times B$\nomenclature{$A\times
  B$}{Product of types}.

As for functions, dependent type theory allows the second type to
depend on the first type. Thus, the type of pairs where the first
element $x$ is in type $A$ and the second element $y$ is in type $B\,
x$ is called ``dependent sum over $B$'' (or sigma-type over $B$), noted $\sumD
a A {B\, a}$\nomenclature{$\sumD a A {B\, a}$}{Dependent sum over
  $B$}.

The introduction and elimination rules are:

\begin{center}
  \AxiomC{$\Gamma,x:A\vdash B:\Type$}
  \AxiomC{$\Gamma\vdash x:A$}
  \AxiomC{$\Gamma\vdash b:B[a/x]$}
  \RightLabel{$\sum$-\textsc{intro}}
  \TrinaryInfC{$\Gamma\vdash (a,b):\sumD x a B$}
  \DisplayProof
  \vspace{1em}

  \AxiomC{$\Gamma\vdash p:\sumD x A B$}
  \RightLabel{$\sum$-$\textsc{intro}_1$}
  \UnaryInfC{$\Gamma\vdash \pi_1 p: A$}
  \DisplayProof
  \qquad
  \AxiomC{$\Gamma\vdash p:\sumD x A B$}
  \RightLabel{$\sum$-$\textsc{intro}_2$}
  \UnaryInfC{$\Gamma\vdash \pi_2 p: B[\pi_1 p/x]$}
  \DisplayProof
\end{center}

\begin{rmq}
  We note that the terminology might be confusing: the dependent
  generalization of products are dependent sums, while dependent
  products are generalization of functions.
\end{rmq}

Under the propositions-as-types principle, $A\times B$ is seen as the
conjonction $A\land B$, and $\sumD x A {B\, a}$ is seen as the
existentially quantified formula $\exists x\, B\, x$.

\subsection{Inductive types}
\label{ssec:inductive}

Dependent type theory actually allows us to define any inductive
type. An inductive type is a type defined only by its introduction
rules, in a free way. Its elimination rules are automatically
determined. We have already seen examples of inductive types: Empty,
Unit, Coproduct.

The most basic example of inductive types might be the type
$\N$\nomenclature{$\N$}{Type of naturals} of naturals. Its
introduction rules are

\begin{center}
  \AxiomC{}
  \RightLabel{$\N$-$\textsc{intro}_0$}
  \UnaryInfC{$0:\N$}
  \DisplayProof
  \qquad
  \AxiomC{$\Gamma\vdash n:\N$}
  \RightLabel{$\N$-$\textsc{intro}_S$}
  \UnaryInfC{$\Gamma \vdash S\, n : \N$}
  \DisplayProof
\end{center}

We also say that its {\em constructors} are $0:\N$ and $S:\N\to
\N$: $\N$ is thus the free monoid generated by $0$ and $S$.
The elimination rule for $\N$ is the famous induction principle

\begin{center}
  \AxiomC{$\Gamma,x:\N\vdash P:\Type$}
  \noLine
  \UnaryInfC{$\Gamma\vdash n:\N$}
  \AxiomC{$\Gamma\vdash c_0:C[0/x]$}
  \noLine
  \UnaryInfC{$\Gamma,n:\N,y:C \vdash c_S : C[S\, n/x]$}
  \RightLabel{$\N$-\textsc{elim}}
  \BinaryInfC{$\Gamma\vdash \mathrm{nat\_ind}(C,c_0,c_S(n,y),n) : C[n/x]$}
  \DisplayProof
\end{center}

This elimination rule allows us to define basic operators on natural
numbers: addition, multiplication, order, \etc{}

\subsection{Paths type}
\label{ssec:path}
One of the most powerful tool in dependent type theory might be the
identity types. They allow us to talk about propositional equality
between inhabitants of a type. The identity type over $A$ will be
noted $a=_A b$ or $a=b$ if $A$ can be inferred from context
\nomenclature{$a=_A b$}{Type of paths from $a$ to $b$ in
  $A$}\nomenclature{$a=b$}{Type of paths from $a$ to $b$}.

$1$ or $\idpath$ or $1_x$ or $\idpath_x$: constructor of
$x=x$\nomenclature{$1_x$ or $1$}{Constant path over
  $x$}\nomenclature{$\idpath_x$ or $\idpath$}{Constant path over $x$}

\begin{center}
  \AxiomC{$\Gamma\vdash a:A$}
  \RightLabel{$=$-\textsc{intro}}
  \UnaryInfC{$\Gamma\vdash \idpath_a : a =_A a$}
  \DisplayProof
  \vspace{1em}

  \AxiomC{$\Gamma,x:A\vdash P:\Type$}
  \AxiomC{$\Gamma\vdash a,b:A$}
  \AxiomC{$\Gamma\vdash p:a=_A b$}
  \AxiomC{$\Gamma\vdash w:P[a/x]$}
  \RightLabel{$=$-\textsc{elim}}
  \QuaternaryInfC{$\Gamma\vdash \transport_P^p(w) : P[b/x]$}
  \DisplayProof
\end{center}

\subsection{Summary}
\label{ssec:mltt_summary}

We can summarize the situation in the following array:

\renewcommand{\arraystretch}{2}
\begin{tabular}{|l|l|l|}
  \hline
  Name & Notation & Proposition-as-types \\
  \hline\hline
  Empty & $\displaystyle{\zero}$ & $\displaystyle{\bot}$ \\
  \hline
  Unit & $\displaystyle{\one}$ & $\displaystyle{\top}$ \\
  \hline
  Coproduct, sum & $\displaystyle{A+B}$ & $\displaystyle{A\lor B}$ \\
  \hline
  Function & $\displaystyle{A\times B}$ & $\displaystyle{A\To B}$ \\
  \hline
  Dependent function, pi-type & $\displaystyle{\prodD x A {B\, x}}$ & $\displaystyle{\forall x,\,
                                                       B\, x}$ \\
  \hline
  Product & $\displaystyle{A\times B}$ & $\displaystyle{A\land B}$ \\
  \hline
  Dependent sum, sigma-type & $\displaystyle{\sumD x A {B\, x}}$ & $\displaystyle{\exists x,\, B\,
                                                    x}$ \\
  \hline
\end{tabular}
\renewcommand{\arraystretch}{1}

\section{Univalence axiom}
\label{sec:ua}

\section{Higher Inductives Types}
\label{sec:hit}

\subsection{Truncations}
\label{ssec:trunc}

$\|A\|_n$: $n$-truncation of $A$%
  \nomenclature{$\Vert\cdot\Vert$}{$n$-truncation of types}

\nomenclature{$\vert\cdot\vert_n$}{$n$-truncation of terms or arrows}

\section{Introduction to homotopy type theory}
\label{sec:hott}
