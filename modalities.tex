\chapter{Higher modalities}
\label{chap:modalities}

As said in the introduction, the main purpose of our work is to build,
from a model $\mathfrak M$ of homotopy type theory, another model
$\mathfrak M'$ satisfying new principles. Of course, $\mathfrak M'$
should be describable {\em inside} $\mathfrak M$. In set theory, it
corresponds to building {\em inner models} (\cite{kunen}).%
%
In type theory, it can be rephrased in terms of left-exact modalities:
it consists of an operator $\modal$ on types such that for any type
$A$, $\modal A$ satisfies a desired property. If the operator has a
``good'' behaviour, then it is a modality, and the universe of all
types satisfying the chosen property forms a new model of homotopy
type theory.

\section{Modalities}
\label{sec:modalities}

\begin{defi}
  \label{def:modality}
  A left exact modality is the data of
  \begin{enumerate}[(i)]
  \item A predicate $P:\Type \to \HProp$
  \item For every type $A$, a type
    $\modal A$ such that $P(\modal A)$
  \item For every type $A$, a map $\eta_A:A \to
    \modal A$
  \end{enumerate}
  such that
  \begin{enumerate}[(i)]
    \setcounter{enumi}{3}
  \item For every types $A$ and $B$, if $P(B)$ then
    \[ \left\{
        \begin{array}{rcl}
          (\modal A \to B) & \to & (A \to B) \\
          f & \mapsto & f \circ \eta_A
        \end{array} \right. \] %
    is an equivalence.
  \item for any $A:\Type$ and $B:A \to \Type$ such that $P(A)$
    and $\prod_{x:A} P(B x)$, then $P\left( \sum_{x:A} B(x)\right)$
  \item for any $A:\Type$ and $x,y:A$, if $\modal A$ is
    contractible, then $\modal (x=y)$ is contractible.
  \end{enumerate}
  Conditions (i) to (iv) define a {\em reflective subuniverse}, (i) to
  (v) a {\em modality}.
\end{defi}

If $\modal$ is a modality, the type of modal types will be denoted
$\Type^\modal$. Let us fix a left-exact modality $\modal$ for the rest
of this section. A modality acts functorialy on $\Type$, in the sense
that

\begin{lem}[Functoriality of modalities]
  Let $A,B:\Type$ and $f:A\to B$. Then there is a map $\modal f:\modal
  A \to \modal B$ such that 
  \begin{itemize}
  \item $\modal f \circ \eta_A = \eta_B \circ f$
  \item if $g:B\to C$, $\modal (g \circ f) = \modal g \circ \modal f$
  \item if $\IsEquiv f$, then $\IsEquiv \modal f$.
  \end{itemize}
\end{lem}

\begin{prop}\label{prop:mod_prop}
  Any left-exact modality $\modal$ satisfies the following
  properties\footnote{Properties needing only a reflective subuniverse
    are annoted by (R), a modality by (M), a left-exact modality by (L)}.
  \begin{itemize}
  \item[\labelitemi(R)] $A$ is modal if and only if $\eta_A$ is an equivalence.
  \item[\labelitemi(R)] $\one$ is modal.
  \item[\labelitemi(R)] $\Type^\modal$ is closed under dependent
    products, \ie{} $\prodD x A {B\, x}$ is modal as soon as all $B\,
    x$ are modal.
  \item[\labelitemi(R)] For any types $A$ and $B$, the map
    \[ \modal(A\times B) \to \modal A \times\modal B \]
    is an equivalence.
  \item[\labelitemi(R)] If $A$ is modal, then for all $x,y:A$, $(x=y)$
    is modal.
  \item[\labelitemi(M)] For every type $A$ and $B:\modal(A)\to\Type^\modal$, then
    \[ \fonction{-\circ\eta_A}{\prodD z {\modal A} {B\, x}}{\prodD a
        A {B(\eta_A\, a)}}{f}{f\circ \eta_A} \]
    is an equivalence.
  \item[\labelitemi(L)] If $A:\Type_n$, then $\modal A:\Type_n$.
  \item[\labelitemi(L)] $\Type^\modal$ is itself modal.
  \item[\labelitemi(L)] If $X,Y:\Type$ and $f:X\to Y$, then the map
    \[ \modal \left( \fib f y\right) \to \fib{\modal f}{\eta_B
        y} \]
    is an equivalence, and the following diagram commutes
\[ \xymatrix{
  \fib f y \ar[r]^\eta \ar[d]_\gamma & \modal \left(\fib f y \right) \ar[dl]\\
  \fib{\modal f}{\eta_B y} & }\] 
  \end{itemize}
\end{prop}

\begin{proof}
  We only prove the last point.
  It is straighforward to define a map
  \[ \phi:\sumD x X  {f x = y}\to
    \sumD x {\modal X} {\modal f x = \eta_Y y},\]
  using $\eta$ functions.
  We will use the following lemma to prove that the function induced
  by $\phi$ defines an equivalence:
  \begin{lem}
    Let $X:\Type$, $Y:\Type^\modal$ and $f:X\to Y$. If for all $y:Y$,
    $\modal (\fib f y)$ is contractible, then $\modal X \simeq Y$.
  \end{lem}
  % 
  Hence we just need to check that every $\modal$-fiber $\modal(\fib \phi {x;p})$ is contractible.
  Technical transformations allow one to prove
  \[ \fib\phi{x;p} \simeq \fib s {y;p^{-1}}\]
  for some $a:\modal X$, $b:\modal Y$ and 
  \[ s:\fib{\eta_X}a \to \fib{\eta_Y}b.\]
  % 
  From definition of left exactness, one can deduce the following:
  \begin{lem}
    Let $A,B:\Type$. Let $f:A\to B$. If $\modal A$ and $\modal B$ are
    contractible, then so is $\fib f b$ for any $b:B$.
  \end{lem}
  Thus, we just need to prove that $\modal(\fib {\eta_X} a)$ and
  $\modal(\fib {\eta_Y} b)$ are contractible. But one can check that
  $\eta$ maps always satisfy this property.
  Finally, $\modal(\fib s{y;p^{-1}})$ is contractible, so $\modal(\fib \phi {x;p})$ also, and the result is proved.
  % \kq{Finish that}

  % \nt{Indeed, I can't follow the proof for the moment}
\end{proof}

% %\kq{Is the following usefull?}
% In the same way, left exact modalities preserve homotopy types.
% \begin{prop}
%   Let $k \leq n$.
%   If $P:\Type_k$, then $\modal \widehat P : \Type_k$, where $\widehat P$
%   is $P$ seen as a $n$-type.
% \end{prop}
% \begin{proof}
%   An $n$-truncated type $P$ can equivalently be described as a type for
%   which the unique map to $\one$ is with $n$-truncated fibers. Thus, the
%   property is a direct corollary of
%   Proposition~\ref{sec:defin-basic-prop} and the fact that $\modal \one =
%   \one$.
% \end{proof}
% \end{proof}

\section{Examples of modalities}
\label{sec:modalities-examples}

\subsection{The identity modality}
\label{ssec:id_mod}

Let us begin with the most simple modality one can imagine: the one
doing nothing. We can define it by letting $\modal A \defeq A$ for any type
$A$, and $\eta_A \defeq \idmap$. Obviously, the desired computation
rules are satisfied, so that the identity modality is indeed a
left-exact modality.

It might sound useless to consider such a modality, but it can be
precious when looking for properties of modalities: if it does not
hold for the identity modality, it cannot hold for an abstract one.


\subsection{Truncations}
\label{ssec:truncations}

The first class of non-trivial examples might be the {\em truncations}
modalities.

\subsection{Double negation modality}
\label{ssec:notnot}

The double negation modality $\modal A \defeq \lnot\lnot A$ is a
modality. Unfortunately, it appears that every type is collapsed to an
$\HProp$, thus it cannot be used as-is. The main purpose of this
thesis, in particular chapter~\ref{chap:sheaf} is to extend this
modality into a better one.

\todo[inline]{Finish modalities}
\section{New type theories}
\label{sec:new-type-theories}

\begin{prop}\label{prop:consistent}
  
\end{prop}

\section{Truncated modalities}
\label{sec:trunc_modalities}



\section{Translation}
\label{sec:translation}



