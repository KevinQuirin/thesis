\chapter{Higher modalities}
\label{chap:modalities}

As said in the introduction, the main purpose of our work is to build,
from a model $\mathfrak M$ of homotopy type theory, another model
$\mathfrak M'$ satisfying new principles. Of course, $\mathfrak M'$
should be describable {\em inside} $\mathfrak M$. In set theory, it
corresponds to building {\em inner models} (\cite{kunen}).%
%
In type theory, it can be rephrased in terms of left-exact modalities:
it consists of an operator $\modal$ on types such that for any type
$A$, $\modal A$ satisfies a desired property. If the operator has a
``good'' behaviour, then it is a modality, and the universe of all
types satisfying the chosen property forms a new model of homotopy
type theory.

\section{Modalities}
\label{sec:modalities}

\begin{defi}
  \label{def:modality}
  A left exact modality is the data of
  \begin{enumerate}[(i)]
  \item A predicate $P:\Type \to \HProp$
  \item For every type $A$, a type
    $\modal A$ such that $P(\modal A)$
  \item For every type $A$, a map $\eta_A:A \to
    \modal A$
  \end{enumerate}
  such that
  \begin{enumerate}[(i)]
    \setcounter{enumi}{3}
  \item For every types $A$ and $B$, if $P(B)$ then
    \[ \left\{
        \begin{array}{rcl}
          (\modal A \to B) & \to & (A \to B) \\
          f & \mapsto & f \circ \eta_A
        \end{array} \right. \] %
    is an equivalence.
  \item for any $A:\Type$ and $B:A \to \Type$ such that $P(A)$
    and $\prod_{x:A} P(B x)$, then $P\left( \sum_{x:A} B(x)\right)$
  \item for any $A:\Type$ and $x,y:A$, if $\modal A$ is
    contractible, then $\modal (x=y)$ is contractible.
  \end{enumerate}
  Conditions (i) to (iv) define a {\em reflective subuniverse}, (i) to
  (v) a {\em modality}.
\end{defi}

\section{Examples of modalities}
\label{sec:modalities-examples}

\subsection{The identity modality}
\label{ssec:id_mod}

Let us begin with the most simple modality one can imagine: the one
doing nothing. We can define it by letting $\modal A \defeq A$ for any type
$A$, and $\eta_A \defeq \idmap$. Obviously, the desired computation
rules are satisfied, so that the identity modality is indeed a
left-exact modality.

It might sound useless to consider such a modality, but it can be
precious when looking for properties of modalities: if it does not
hold for the identity modality, it cannot hold for an abstract one.


\subsection{Truncations}
\label{ssec:truncations}

The first class of non-trivial examples might be the {\em truncations}
modalities.

\subsection{Double negation modality}
\label{ssec:notnot}

The double negation modality $\modal A \defeq \lnot\lnot A$ is a
modality. Unfortunately, it appears that every type is collapsed to an
$\HProp$, thus it cannot be used as-is. The main purpose of this
thesis, in particular chapter~\ref{chap:sheaf} is to extend this
modality into a better one.

\section{New type theories}
\label{sec:translation}

\section{Truncated modalities}
\label{sec:trunc_modalities}

