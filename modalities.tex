\chapter{Higher modalities}
\label{chap:modalities}

As said in the introduction, the main purpose of our work is to build,
from a model $\mathfrak M$ of homotopy type theory, another model
$\mathfrak M'$ satisfying new principles. Of course, $\mathfrak M'$
should be describable {\em inside} $\mathfrak M$. In set theory, it
corresponds to building {\em inner models} (\cite{kunen}).%
%
In type theory, it can be rephrased in terms of left-exact modalities:
it consists of an operator $\modal$ on types such that for any type
$A$, $\modal A$ satisfies a desired property. If the operator has a
``good'' behaviour, then it is a modality, and the universe of all
types satisfying the chosen property forms a new model of homotopy
type theory.

\section{Modalities}
\label{sec:modalities}

\begin{defi}
  \label{def:modality}
  A left exact modality is the data of
  \begin{enumerate}[(i)]
  \item A predicate $P:\Type \to \HProp$
  \item For every type $A$, a type
    $\modal A$ such that $P(\modal A)$
  \item For every type $A$, a map $\eta_A:A \to
    \modal A$
  \end{enumerate}
  such that
  \begin{enumerate}[(i)]
    \setcounter{enumi}{3}
  \item For every types $A$ and $B$, if $P(B)$ then
    \[ \left\{
        \begin{array}{rcl}
          (\modal A \to B) & \to & (A \to B) \\
          f & \mapsto & f \circ \eta_A
        \end{array} \right. \] %
    is an equivalence.
  \item for any $A:\Type$ and $B:A \to \Type$ such that $P(A)$
    and $\prod_{x:A} P(B x)$, then $P\left( \sum_{x:A} B(x)\right)$
  \item for any $A:\Type$ and $x,y:A$, if $\modal A$ is
    contractible, then $\modal (x=y)$ is contractible.
  \end{enumerate}
  Conditions (i) to (iv) define a {\em reflective subuniverse}, (i) to
  (v) a {\em modality}.
\end{defi}

If $\modal$ is a modality, the type of modal types will be denoted
$\Type^\modal$. Let us fix a left-exact modality $\modal$ for the rest
of this section. A modality acts functorialy on $\Type$, in the sense
that

\begin{lem}[Functoriality of modalities]
  Let $A,B:\Type$ and $f:A\to B$. Then there is a map $\modal f:\modal
  A \to \modal B$ such that 
  \begin{itemize}
  \item $\modal f \circ \eta_A = \eta_B \circ f$
  \item if $g:B\to C$, $\modal (g \circ f) = \modal g \circ \modal f$
  \item if $\IsEquiv f$, then $\IsEquiv \modal f$.
  \end{itemize}
\end{lem}

\begin{prop}\label{prop:mod_prop}
  Any left-exact modality $\modal$ satisfies the following
  properties\footnote{Properties needing only a reflective subuniverse
    are annoted by (R), a modality by (M), a left-exact modality by (L)}.
  \begin{itemize}
  \item[\labelitemi(R)] $A$ is modal if and only if $\eta_A$ is an equivalence.
  \item[\labelitemi(R)] $\one$ is modal.
  \item[\labelitemi(R)] $\Type^\modal$ is closed under dependent
    products, \ie{} $\prodD x A {B\, x}$ is modal as soon as all $B\,
    x$ are modal.
  \item[\labelitemi(R)] For any types $A$ and $B$, the map
    \[ \modal(A\times B) \to \modal A \times\modal B \]
    is an equivalence.
  \item[\labelitemi(R)] If $A$ is modal, then for all $x,y:A$, $(x=y)$
    is modal.
  \item[\labelitemi(M)] For every type $A$ and $B:\modal(A)\to\Type^\modal$, then
    \[ \fonction{-\circ\eta_A}{\prodD z {\modal A} {B\, x}}{\prodD a
        A {B(\eta_A\, a)}}{f}{f\circ \eta_A} \]
    is an equivalence.
  \item[\labelitemi(M)] If $A,B:\Type$ are modal, then so are $\IsType
    n A$, $A\simeq B$ and $\IsEquiv f$ for all $f:A\to B$.
  \item[\labelitemi(L)] If $A:\Type_n$, then $\modal A:\Type_n$.
  \item[\labelitemi(L)] $\Type^\modal$ is itself modal.
  \item[\labelitemi(L)] If $X,Y:\Type$ and $f:X\to Y$, then the map
    \[ \modal \left( \fib f y\right) \to \fib{\modal f}{\eta_B
        y} \]
    is an equivalence, and the following diagram commutes
\[ \xymatrix{
  \fib f y \ar[r]^\eta \ar[d]_\gamma & \modal \left(\fib f y \right) \ar[dl]\\
  \fib{\modal f}{\eta_B y} & }\] 
  \end{itemize}
\end{prop}

\begin{proof}
  We only prove the last point.
  It is straighforward to define a map
  \[ \phi:\sumD x X  {f x = y}\to
    \sumD x {\modal X} {\modal f x = \eta_Y y},\]
  using $\eta$ functions.
  We will use the following lemma to prove that the function induced
  by $\phi$ defines an equivalence:
  \begin{lem}
    Let $X:\Type$, $Y:\Type^\modal$ and $f:X\to Y$. If for all $y:Y$,
    $\modal (\fib f y)$ is contractible, then $\modal X \simeq Y$.
  \end{lem}
  % 
  Hence we just need to check that every $\modal$-fiber $\modal(\fib \phi {x;p})$ is contractible.
  Technical transformations allow one to prove
  \[ \fib\phi{x;p} \simeq \fib s {y;p^{-1}}\]
  for
  \[
    \fonction{s}{\fib{\eta_X}x}{\fib{\eta_Y}{\modal f\, x}}{(a,q)}{(f\, a,-)}
  \]
  But left-exctness allows to characterize the contractibility of fibers:
  \begin{lem}
    Let $A,B:\Type$. Let $f:A\to B$. If $\modal A$ and $\modal B$ are
    contractible, then so is $\fib f b$ for any $b:B$.
  \end{lem}
  Thus, we just need to prove that $\modal(\fib {\eta_X} a)$ and
  $\modal(\fib {\eta_Y} b)$ are contractible. But one can check that
  $\eta$ maps always satisfy this property.
  Finally, $\modal(\fib s{y;p^{-1}})$ is contractible, so $\modal(\fib \phi {x;p})$ also, and the result is proved.
  % \kq{Finish that}

  % \nt{Indeed, I can't follow the proof for the moment}
\end{proof}

% %\kq{Is the following usefull?}
% In the same way, left exact modalities preserve homotopy types.
% \begin{prop}
%   Let $k \leq n$.
%   If $P:\Type_k$, then $\modal \widehat P : \Type_k$, where $\widehat P$
%   is $P$ seen as a $n$-type.
% \end{prop}
% \begin{proof}
%   An $n$-truncated type $P$ can equivalently be described as a type for
%   which the unique map to $\one$ is with $n$-truncated fibers. Thus, the
%   property is a direct corollary of
%   Proposition~\ref{sec:defin-basic-prop} and the fact that $\modal \one =
%   \one$.
% \end{proof}
% \end{proof}

\section{Examples of modalities}
\label{sec:modalities-examples}

\subsection{The identity modality}
\label{ssec:id_mod}

Let us begin with the most simple modality one can imagine: the one
doing nothing. We can define it by letting $\modal A \defeq A$ for any type
$A$, and $\eta_A \defeq \idmap$. Obviously, the desired computation
rules are satisfied, so that the identity modality is indeed a
left-exact modality.

It might sound useless to consider such a modality, but it can be
precious when looking for properties of modalities: if it does not
hold for the identity modality, it cannot hold for an abstract one.


\subsection{Truncations}
\label{ssec:truncations}

The first class of non-trivial examples might be the {\em truncations}
modalities, seen in~\ref{ssec:trunc}.

\subsection{Double negation modality}
\label{ssec:notnot}

The double negation modality $\modal A \defeq \lnot\lnot A$ is a
modality. Unfortunately, it appears that every type is collapsed to an
$\HProp$, thus it cannot be used as-is. The main purpose of this
thesis, in particular chapter~\ref{chap:sheaf} is to extend this
modality into a better one.

\todo[inline]{Finish modalities}
\section{New type theories}
\label{sec:new-type-theories}

\begin{prop}\label{prop:consistent}
  A left exact modality $\modal$ induces a consistent type theory if
  and only if $\modal \zero$ can not be inhabited in the initial type
  theory. In that case, the modality is said to be consistent.
\end{prop}
\begin{proof}
  By condition (iv) of Definition~\ref{def:modality},
  $\modal \zero$ is an initial object of $\Type^\modal$, and thus
  corresponds to false for modal mere proposition.
  % 
  As $\modal \one = \one$, $\Type^\modal$ is consistent when
  $\modal \zero \neq \one$, that is when there is no proof of
  $\modal \zero$.
\end{proof}

\section{Truncated modalities}
\label{sec:trunc_modalities}

As for colimits, we define a truncated version of modalities, in order
to use it in chapter~\ref{chap:sheaf}. Basically, a truncated modality
is the same as a modality, but restricted to $\Type_n$. 

\begin{defi}[Truncated modality]
  \label{def:tr_mod}
  Let $n\geq -1$ be a truncation index. A left exact modality at level
  $n$ is the data of
  \begin{enumerate}[(i)]
  \item A predicate $P:\Type_n \to \HProp$
  \item For every $n$-truncated type $A$, a $n$-truncated type
    $\modal A$ such that $P(\modal A)$
  \item For every $n$-truncated type $A$, a map $\eta_A:A \to
    \modal A$
  \end{enumerate}
  such that
  \begin{enumerate}[(i)]
    \setcounter{enumi}{3}
  \item For every $n$-truncated types $A$ and $B$, if $P(B)$ then
    \[\left\{
      \begin{array}{rcl}
        (\modal A \to B) & \to & (A \to B) \\
        f & \mapsto & f \circ \eta_A
      \end{array} \right.\]
    is an equivalence.
  \item for any $A:\Type_n$ and $B:A \to \Type_n$ such that $P(A)$
    and $\prod_{x:A} P(B x)$, then $P\left( \sum_{x:A} B(x)\right)$
  \item for any $A:\Type_n$ and $x,y:A$, if $\modal A$ is
    contractible, then $\modal (x=y)$ is contractible.
  \end{enumerate}
\end{defi}

Properties of truncated left-exact modalities described
in~\ref{prop:mod_prop} are still true when restricted to $n$-truncated
types, except the one that does not make sense: $\Type_n^\modal$
cannot be modal, as it is not even a $n$-truncated type.


\section{Translation}
\label{sec:translation}

As said in section~\ref{sec:new-type-theories}, left-exact modalities
allows to perform model transformation. But it can be enhanced a bit
by exhibiting a {\em translation} of type theories, as it has been done for
forcing~\cite{jaber2012extending}. Let us explain
here how this translation works.

Let $\modal$ be a left-exact modality. We describe, for each type
constructor, how to build its translation. We denote $\pi_\modal(A)$
the proof that $\modal(A)$ is always modal.

\begin{itemize}
\item For types
\[
\begin{array}{lcl}
  \left[ \Type\right] &\defeq& (\Type^\modal,\pi_{\Type^\modal})
\end{array}
\]
where $\pi_{\Type^\modal}$ is a proof that $\Type^\modal$ is itself
modal.
To ease the reading in what follows, we introduce the notation  \[ 
  \Lbrack A \Rbrack \defeq \pi_1 \left[ A \right]\]

\item For dependent sums
\[
\begin{array}{lcl}
\left[ \sumD x A B \right] &\defeq&  \left( \sumD x{\Lbrack A \Rbrack}
                                  {\Lbrack B\Rbrack} , \pi_\Sigma^{[A],[B]}
                                \right)\\[0.5em]
  \left[  (x,y)\right] &\defeq& ([x],[y]) \\[0.5em]
  \left[  \pi_i t\right] &\defeq& \pi_i [t] \\[0.5em]
\end{array}
\]
where $\pi_{\Sigma}^{A,B}$ is a proof that $\sumD x A B$ is modal when
$A$ and $B$ are.
\item For dependent products
\[
\begin{array}{lcl}
 \left[ \prod_{x:A} B \right] &\defeq& \left( \prod_{x:\Lbrack A \Rbrack}
                                   \Lbrack B\Rbrack , \pi_{\Pi}^{[A],[B]}
                                  \right)\\[0.5em]
\left[  \lambda\, x:A,~M \right] &\defeq&\lambda\,x:\Lbrack A
                                     \Rbrack,~[ M ]
  \\[0.5em]
  \left[ t \, t' \right] &\defeq& [t] [t'] \\[0.5em]
\end{array}
\]
where $\pi_{\Pi}^{A,B}$ is a proof that $\prodD x A B$ is modal when $B$ is.
\item For paths
\[
\begin{array}{lcl}
\left[  A=B \right] &\defeq& \left( [A] = [ B] , \pi_=^{[A],[B]}
                             \right)\\[0.5em]
\left[ 1 \right] &\defeq& 1\\[0.5em]
\left[ J \right] &\defeq& J \\[0.5em]
\end{array}
\]
where $\pi_{=}^{A,B}$ is a proof that $A=B$ is modal when
$A$ and $B$ are, if $A,B:\Type$, or a proof that $A=B$ is modal as
soon as their type is modal if $A,B:X$.
\item For positive types (we only treat the case of the sum as an example)
\[
\begin{array}{lcl}
\left[  A+B \right] &\defeq& \left( \modal(\Lbrack A \Rbrack + \Lbrack B
                        \Rbrack); \pi_\modal(\Lbrack A \Rbrack + \Lbrack B
                        \Rbrack)\right)\\[0.5em]
\left[  \mathrm{in}_\ell t \right] &\defeq& \eta (\mathrm{in}_\ell [t]) \\[0.5em]
\left[  \mathrm{in}_r t \right] &\defeq& \eta (\mathrm{in}_r [t]) \\[0.5em]
\left[ \langle f ,g\rangle\right] &\defeq& \modal_{\mathrm{rec}}^{\Lbrack A\Rbrack +
                                      \Lbrack B\Rbrack} \langle
                                      [f],[g]\rangle\\[0.5em]
\end{array}
\]
\item For truncations ($i\leqslant n$)
\[
\begin{array}{lcl}
  \left[  \|A\|_i \right] &\defeq& (\modal \| \Lbrack A\Rbrack  \|_i;
                                   \pi_\modal(\| \Lbrack A\Rbrack
                                   \|_i)) \\[0.5em]
  \left[ |t|_i \right] &\defeq& \eta |[t]|_i \\[0.5em]
  \left[ |f|_i \right] &\defeq& \modal_{\mathrm{rec}}^{\| \Lbrack
                                  A\Rbrack  \|_i} | [f] |_i
\end{array}
\]

\end{itemize}

As in the forcing translation~\cite{jaber2012extending}, the main
issue is that convertibility . For
example, let $f:A \to X$ and $g:B\to X$. Then $\langle f,g\rangle
(\mathrm{in}_\ell t)$


Two solutions to this problem can come to our mind:
\begin{itemize}
\item We can use the eliminator $J$ of equality in the conversion
  rule, but this would require to use a type theory with explicit
  conversion in the syntax, like in~\cite{jaber2012extending}.
\item We can ask the modality $\modal$ to be a strict modality, in the
  sense that retraction in the equivalence of $(-\circ\eta)$ is conversion
  instead of equality, like for the closed modality. 
  That way, we keep the conversion rule, \ie{} if $\Gamma \vdash
  u \equiv v$, then $\Lbrack \Gamma \Rbrack \vdash [u] \equiv [v]$.
  That is the solution we chose.
\end{itemize}

The usual result we want about any translation is its soundness:
\begin{prop}[Soundness of the translation]
  Let $\Gamma$ is a valid context, $A$ a type and $t$ a term.
  If $\Gamma \vdash t : A$, then 
  \[\Lbrack \Gamma \Rbrack \vdash [t] : \Lbrack A \Rbrack. \]
\end{prop}

\begin{rmq}
  We note that if the modality is not left-exact, like truncations
  modalities, then $\Type^\modal$ is not itself modal. It is although
  still possible to write a translation, but we can only define it on
  a type theory with only one universe.
\end{rmq}