\documentclass[a4paper, 10pt]{memoir}

\let\leftbar\undefined
\let\endleftbar\undefined
\usepackage[nothm]{thmbox}
\let\proof\undefined\let\endproof\undefined

\usepackage[T1]{fontenc}
\usepackage[french,english]{babel}
\selectlanguage{english}
\usepackage[utf8]{inputenc}
\usepackage{mathbbol} %permet \mathbb{1} \Lbrack \Rbrack ...
\usepackage{amsmath,amssymb,amsthm,amsfonts} %packages AMS
\usepackage{wasysym}
\usepackage{calrsfs} %\mathcal{} plus rondes
\usepackage{enumerate} %permet de changer le type de numérotation
                       %dans enumerate
\usepackage{graphicx} %permet d'afficher des images
\usepackage{titlesec,titletoc} %modifie les titres et la table des matières
\usepackage{lettrine} %permet les lettrines
\usepackage{todonotes}
\PassOptionsToPackage{hyphens}{url}
\usepackage[style=alphabetic,backref=true]{biblatex}
\renewbibmacro{in:}{}
\usepackage[colorlinks=true,linkcolor=black,urlcolor=blue,citecolor=gray]{hyperref}
\usepackage{csquotes}
\bibliography{biblio}
\usepackage{mathpazo}
\linespread{1.05}
\usepackage{lipsum}
\setlipsumdefault{1-4}


\usepackage{adjustbox}
\usepackage{graphicx}




\tolerance 1414
\hbadness 1414
\emergencystretch 1.5em
\hfuzz 0.3pt
\widowpenalty=10000
\vfuzz \hfuzz
\raggedbottom

\newtheoremstyle{theoreme}
 {2em} %espace avant le théorème
 {2em} %espace après le théorème
 {\slshape} %fonte du texte
 {} %indentation
 {\bfseries} %fonte du titre
 {} %ponctuation après le titre
 {\newline} %espace après le titre
 {\thmname{#1}\thmnumber{ #2}\thmnote{~:~#3}} %forme du titre


\theoremstyle{theoreme} 
\newtheorem{thm}{Theorem}[chapter]
\newtheorem{cor}[thm]{Corollary}
\newtheorem{prop}[thm]{Proposition}
\newtheorem{defi}[thm]{Definition}
\newtheorem{lem}[thm]{Lemma}
\newtheorem{cond}[thm]{Condition}
\newtheorem{ax}[thm]{Axiom}

\newtheorem{thmfr}{Théoreme}[chapter]
\newtheorem{corfr}[thm]{Corollaire}
\newtheorem{defifr}[thm]{Définition}
\newtheorem{lemfr}[thm]{Lemme}
\newtheorem{axfr}[thm]{Axiome}

\newenvironment{prooflem}%
    {\renewcommand{\qedsymbol}{$\Diamond$}%
     \begin{proof}}%
    {\end{proof}%
     \renewcommand{\qedsymbol}{$\square$}}

% \newenvironment{rmq}% 
% 	{\noindent \textsc{Nota}~\textendash~}%
% 	{\medskip}

% \newenvironment{exm}%
% 	{\noindent \textsc{Example}~\textendash~}%
% 	{\medskip}

\newenvironment{rmqs}%
	{\noindent \textsc{Notae}~:~ \medskip%
	 \begin{list}%
		{\textendash}%
		{\setlength{\itemsep}{2ex}}}%
	{\end{list} \medskip}

\newenvironment{exms}%
	{\noindent \textsc{Examples}~:~ \medskip%
	 \begin{list}%
		{\textendash}%
		{\setlength{\itemsep}{2ex}}}%
	{\end{list} \medskip}

\newenvironment{nota}% 
	{\noindent \textsc{Notation}~\textendash~}%
	{\medskip}

%% Maths %%

        % HoTT %
\usepackage[all]{xy}
\def\dar[#1]#2{\ar@<-#2>[#1]\ar@<#2>[#1]} %double arrows in xy
\def\tar[#1]#2{\ar@<#2>[#1]\ar@<0pt>[#1]\ar@<-#2>[#1]} %triple arrows in xy
\DeclareMathOperator{\Type}{Type}
\DeclareMathOperator{\HProp}{HProp}
\DeclareMathOperator{\HSet}{HSet}
\DeclareMathOperator{\IsHProp}{IsHProp}
\DeclareMathOperator{\IsHSet}{IsHSet}
\DeclareMathOperator{\nat}{nat}
\DeclareMathOperator{\Unit}{Unit}
\DeclareMathOperator{\im}{Im}
\DeclareMathOperator{\id}{id}
\DeclareMathOperator{\Contr}{Contr}
\DeclareMathOperator{\IsContr}{IsContr}
\DeclareMathOperator{\IsEquiv}{IsEquiv}
\DeclareMathOperator{\precompose}{\mathrm{precompose}}
\DeclareMathOperator{\postcompose}{\mathrm{postcompose}}
\DeclareMathOperator{\idmap}{\mathrm{idmap}}
\DeclareMathOperator{\cocone}{cocone}
\DeclareMathOperator{\inl}{inl}
\DeclareMathOperator{\inr}{inr}
\DeclareMathOperator{\transport}{transport}
\DeclareMathOperator{\tr}{tr}
\DeclareMathOperator{\Sym}{Sym}
\DeclareMathOperator{\Trans}{Trans}
\DeclareMathOperator{\Refl}{Refl}


\newcommand \defeq {\overset{de\hspace{-0.2ex}f}{=}}

\def\mymathhyphen{{\hbox{-}}}

\newcommand{\IsType}[1]
{\mathop{\mathrm{Is\mymathhyphen}#1\mathrm{\mymathhyphen type}} }

\newcommand{\modal}{\ensuremath{\ocircle}}
\newcommand \True {\top}
\newcommand \idpath {\mathrm{idpath}}
\newcommand \False {\bot}
\newcommand \closure[1] {\overline{#1}}
\newcommand \Char[1] {\chi_{#1}}%{\mathrm{char}(#1)}
\newcommand \E {\mathcal{E}}
\newcommand \Hom[1] {\mathrm{Hom}_{#1}}
\newcommand \Obj {\mathrm{Obj}}
\newcommand \Sh[1] {\mathrm{Sh}_{#1}}
\newcommand \squash[1] {\| #1 \| }
\newcommand \separated {\mathop{\square_{n+1}} }
\newcommand \fib[2] {\mathrm{fib}_{#1}(#2)}
\newcommand \colim {\mathrm{colim}}
\newcommand \zero {\mathbf{0}}
\newcommand \one {\mathbf{1}}
\newcommand \unittt{\star}
\newcommand \two {\mathbf{2}}
\newcommand{\sumD}[3]{\sum_{#1:#2}\, #3}
\newcommand{\prodD}[3]{\prod_{#1:#2}\, #3}
\newcommand{\homot}{\sim}
\newcommand{\retr}{\mathrm{retr}}
\newcommand{\sect}{\mathrm{sect}}
\newcommand{\adj}{\mathrm{adj}}
\newcommand{\ap}[1]{\mathrm{ap}_{#1}}
\newcommand{\inv}[1]{#1^{-1}}
\newcommand{\concat}[2]{#1\cdot #2}
\newcommand{\happly}{\mathrm{happly}}
\newcommand{\Sone}{\mathbb{S}^1}
\newcommand{\baseS}{\mathrm{base}}
\newcommand{\loopS}{\mathrm{loop}}
\newcommand{\coeq}[2]{\mathrm{Coeq}^{#1,#2}}

\DeclareMathOperator{\issep}{IsSeparated}
\DeclareMathOperator{\issheaf}{IsSheaf}
\DeclareMathOperator{\KP}{KP}
\DeclareMathOperator{\kp}{kp}

	% Différentielles %
\newcommand{\diff}[1]{\mathrm{d}#1} %élément différentiel
\newcommand{\dpart}[2]{\frac{\partial #1}{\partial #2}} %dérivée partielle d'ordre 1

	% Ensembles %
\newcommand{\N}{\mathbb{N}} %entiers naturels
\newcommand{\R}{\mathbb{R}} %réels
\newcommand{\Q}{\mathbb{Q}} %rationnels
\newcommand{\Z}{\mathbb{Z}} %entiers relatifs
\newcommand{\C}{\mathbb{C}} %complexe
\newcommand{\K}{\mathbb{K}} %corps
\newcommand{\perm}{\mathfrak{S}} %permutations

        % Flèches %
\newcommand{\To}{\Rightarrow}
\newcommand{\ot}{\leftarrow}
\newcommand{\oT}{\Leftarrow}
\newcommand{\tot}{\leftrightarrow}
\newcommand{\ToT}{\Leftrightarrow}
\newcommand{\cto}{\hookrightarrow}
%\newcommand{\cot}{\hookleftarrow}



	% Fonctions %
\newcommand{\fonction}[5]{ %{nom de la fonction}{ensemble de départ}{ensemble d'arrivée}{nom de la variable}{image de la variable}
	#1 \, \colon \left.
	 \begin{array}{ccl}
		#2 & \longrightarrow & #3 \\
		#4 & \longmapsto & #5
	\end{array}
	\right.
}

\def\restriction#1#2{\mathchoice %restriction d'une fonction à un ensemble
              {\setbox1\hbox{${\displaystyle #1}_{\scriptstyle #2}$}
              \restrictionaux{#1}{#2}}
              {\setbox1\hbox{${\textstyle #1}_{\scriptstyle #2}$}
              \restrictionaux{#1}{#2}}
              {\setbox1\hbox{${\scriptstyle #1}_{\scriptscriptstyle #2}$}
              \restrictionaux{#1}{#2}}
              {\setbox1\hbox{${\scriptscriptstyle #1}_{\scriptscriptstyle #2}$}
              \restrictionaux{#1}{#2}}}
\def\restrictionaux#1#2{{#1\,\smash{\vrule height .8\ht1 depth .85\dp1}}_{\,#2}} 

	% Opérateurs %
\DeclareMathOperator{\rk}{Rk} %rang
\DeclareMathOperator{\vect}{Vect} %espace vectoriel engendré
\DeclareMathOperator{\mat}{mat} %matrice
\DeclareMathOperator{\Ker}{Ker} %noyau
\DeclareMathOperator{\diag}{diag} %matrice diagonale
\DeclareMathOperator{\jac}{Jac} %jacobien


\newcommand*{\EnsQ}[2]%
{\ensuremath{%
   \raisebox{.32ex}{\ensuremath{#1}}/\!\raisebox{-.60ex}{\ensuremath{#2}}}}


 % Lettrines %
\renewcommand{\LettrineFontHook}{\fontfamily{yfrak} \selectfont \color[gray]{0.7}}

\newenvironment{fracarray}[1]%
{\renewcommand{\arraystretch}{1.4}%
 \begin{array}{#1}%
}%
{\end{array}%
 \renewcommand{\arraystretch}{1}%
}

\newcommand{\ie}{\emph{i.e.}}
\newcommand{\eg}{\emph{e.g.}}
\newcommand{\etc}{\emph{etc.}}


\newenvironment{rmq}
{\begin{thmbox}[M]{\textsc{Nota}}\noindent}
{\end{thmbox}\medskip}

\newenvironment{exm}
{\begin{thmbox}[M]{\textsc{Example}}\noindent}
{\end{thmbox}\medskip}

\newenvironment{exmfr}
{\begin{thmbox}[M]{\textsc{Exemple}}\noindent}
{\end{thmbox}\medskip}


\usepackage[refpage, intoc]{nomencl}
\makenomenclature

\usepackage{bussproofs}
\setcounter{tocdepth}{2}
\setsecnumdepth{subsection}

\usepackage{nicefrac}

\author{Kevin \textsc{Quirin}}
\title{Lawvere-Tierney Sheafification in Homotopy Type Theory}
\date{2016}

%Chapter Page
\usepackage[T1]{fontenc}
\usepackage{kpfonts}
\setSingleSpace{1.1}
\SingleSpacing
\usepackage{xcolor,calc}
\definecolor{chaptercolor}{gray}{0.8}
% helper macros
\newcommand\numlifter[1]{\raisebox{-2cm}[0pt][0pt]{\smash{#1}}}
\newcommand\numindent{\kern37pt}
\newlength\chaptertitleboxheight
\makechapterstyle{hansen}{
\renewcommand\printchaptername{\raggedleft}
\renewcommand\printchapternum{%
\begingroup%
\leavevmode%
\chapnumfont%
\strut%
\numlifter{\thechapter}%
\numindent%
\endgroup%
}
\renewcommand*{\printchapternonum}{%
\vphantom{\begingroup%
\leavevmode%
\chapnumfont%
\numlifter{\vphantom{9}}%
\numindent%
\endgroup}
\afterchapternum}
\setlength\midchapskip{0pt}
\setlength\beforechapskip{0.5\baselineskip}
\setlength{\afterchapskip}{3\baselineskip}
\renewcommand\chapnumfont{%
\fontsize{4cm}{0cm}%
\bfseries%
\sffamily%
\color{chaptercolor}%
}
\renewcommand\chaptitlefont{%
\normalfont%
\huge%
\bfseries%
\raggedleft%
}%
\settototalheight\chaptertitleboxheight{%
\parbox{\textwidth}{\chaptitlefont \strut bg\\bg\strut}}
\renewcommand\printchaptertitle[1]{%
\parbox[t][\chaptertitleboxheight][t]{\textwidth}{%
%\microtypesetup{protrusion=false}% add this if you use microtype
\chaptitlefont\strut ##1\strut}%
}
}
\chapterstyle{hansen}

%% Titlepage

% \newlength\dropp
% \makeatletter
% \newcommand*{\titleDS}{\begingroup% DS Thesis
% \dropp=0.1\textheight
% %\vspace*{\drop}
% \centering
% {\Large\bfseries Lawvere-Tierney Sheafification in Homotopy Type Theory}\par
% \vspace{0.6\baselineskip}
% {By}\\[0.6\baselineskip]
% {Kevin~Quirin\\[0.6\baselineskip]
% %A Thesis Submitted to the Graduate\\[0.5\baselineskip]
% %Faculty of The University\\[0.5\baselineskip]
% %in Partial Fulfillment of the\\[0.5\baselineskip]
% %Requirements for the Degree of\\[0.5\baselineskip]
% %DEGREE\\[0.5\baselineskip]
% Major Subject: Computer Science}\par
% \flushleft
% {Approved by the \\
% Examining Committee:}\par
% \vspace{1.5\baselineskip}
% \rule{15em}{0.4pt}\\
% Pierre Cointe, Thesis Advisor \\[1\baselineskip]
% \rule{15em}{0.4pt}\\
% Nicolas Tabareau, Thesis Advisor \\[1\baselineskip]
% \rule{15em}{0.4pt}\\
% Carl F. Gau\ss, Göttingen\\[1\baselineskip]
% \rule{15em}{0.4pt}\\
% Euclid, Athens \\[1\baselineskip]
% \centering
% \vspace{1\baselineskip}
% Mines de Nantes \\
% [\baselineskip]
% 2016\par 
% \vfill
% \endgroup}
% \makeatother


\def\pullbackcorner{%
    \ar@{-}[]+R+<6pt,-1pt>;[]+RD+<6pt,-6pt>%
    \ar@{-}[]+D+<1pt,-6pt>;[]+RD+<6pt,-6pt>%
  }

\newcommand{\code}[1]{\texttt{#1}}

\usepackage[a4paper]{geometry}

\definecolor{@doctoratcolor}{RGB}{91, 91, 91}
      \definecolor{@verticalgreen}{RGB}{238, 238, 238}
      \definecolor{@horizontalgray}{gray}{0.97265}
      \definecolor{@textgray}{gray}{0.5}
      \definecolor{@bandeaucolor}{RGB}{190, 190, 190}
      \definecolor{@titlecolor}{RGB}{116, 116, 116}
\newcommand{\Hugecover}{\fontsize{25}{30}\selectfont}
\newcommand{\hugecover}{\fontsize{25}{30}\selectfont}
\newcommand{\LARGEcover}{\fontsize{20}{25}\selectfont}
\newcommand{\Largecover}{\fontsize{17}{22}\selectfont}
\newcommand{\normalsizecover}{\fontsize{12}{14.5}\selectfont}
\newcommand{\smallcover}{\fontsize{11}{13.6}\selectfont}
\newcommand{\footnotesizecover}{\fontsize{10}{12}\selectfont}
\newcommand{\scriptsizecover}{\fontsize{8}{9.5}\selectfont}
\newcommand{\tinysizecover}{\fontsize{6}{7}\selectfont}

\DisemulatePackage{setspace}
\usepackage{setspace}

\makeatletter
\newcommand{\@drawbackground}
   { % Horizontal grey-green box
   \noindent\raisebox{0pt}[0pt][0pt]
                     {\colorbox{@verticalgreen}
                               {\makebox[0.5\textwidth]
                                        {\raisebox{-\textheight}
                                                  {~}}}}

   \vspace{4\baselineskip}

   {
   \noindent\hspace{0.45\textwidth}
            \raisebox{0pt}[0pt][0pt]
                     {\color{@horizontalgray}
                      \rule{0.55\textwidth}
                           {2.5\baselineskip}}
            \hspace{-0.45\textwidth}
   }

   \renewcommand{\sfdefault}{phv}
   \renewcommand{\rmdefault}{ptm}
   \renewcommand{\ttdefault}{pcr}
   \sffamily
   \fontsize{12pt}{14pt}
   \selectfont

   \noindent\makebox[0.55\textwidth][r]
                    {\Hugecover\textbf{\textcolor{@doctoratcolor}
                                                 {Th\`ese de Doctorat}}}

   {
   \noindent\color{@bandeaucolor}
            \rule{0.55\textwidth}
                 {2.5\baselineskip}
            \color{@horizontalgray}
            \rule{0.45\textwidth}
                 {2.5\baselineskip}
   }

   \vspace{3\baselineskip}

   {
   \noindent\hspace{0.45\textwidth}
            \raisebox{0pt}[0pt][0pt]
                     {\color{@horizontalgray}
                      \rule{0.55\textwidth}
                           {2.5\baselineskip}}
            \hspace{-0.45\textwidth}
   }

   \vspace{-15\baselineskip}
   \includegraphics[width=4.5cm]{logos/LUNAM-CUE}
   \hfill
   \hfill
   \hfill
   \ifx\@cosupervisingforeigninstitution\@empty
   \else
   %    \raisebox{\@cosupervisingforeigninstitutionlogovoffset}[0pt][0pt]{\includegraphics[width=\@cosupervisingforeigninstitutionlogowidth]{logos/\@cosupervisingforeigninstitutionlogo}}~
   %    \hfill
   % \fi
   % \ifx\@coinstitution\@empty
   % \else
   % \global\let\@coinstitutionlogovoffset{10pt}
   % \raisebox{\@coinstitutionlogovoffset}[0pt][0pt]{\includegraphics[width=\@coinstitutionlogowidth]{logos/MinesNantes}}~
   % \hfill
   % \fi
   \raisebox{0pt}[0pt][0pt]{\includegraphics[width=2.4cm]{logos/MinesNantes}}~
   \vspace{11\baselineskip}
 }
\makeatother

\makeatletter
\newcommand{\@committee}{%
  \noindent
  \begin{tabular}{ll}
    \footnotesizecover%
    {Rapporteurs}  &  {\textbf{M. Mart\'in Escard\'o}, Reader, University of Birmingham} \\
                   &  {\textbf{M. André Hirschowitz}, Professor, University of Nice} \\
    {Examinateurs} &  {\textbf{M. Pierre Cointe}, Professor, Mines de Nantes} \\
                   &  {\textbf{M. Matthieu Sozeau}, Chargé de
                     recherche, Inria Paris} \\
                   &  {\textbf{M. Bas Spitters}, Associate Professor,
                     University of Aarhus} \\
    {Directeur de thèse} & {\textbf{M. Nicolas Tabareau}, Chargé de
                              recherche, Inria Rennes-Bretagne-Atlantique}
   \end{tabular}}
\makeatother



\makeatletter

%%%%%%%%%%%%%%%%%%%%%%%%%%%
% R'esum'e en franc,ais
%

\newsavebox{\@resumeBox}
\newbox{\@resumeBoxGlobal}

\newenvironment{resume}
   {
   \newgeometry{margin=1.5cm}
   \begin{lrbox}{\@resumeBox}%
      \renewcommand{\sfdefault}{phv}
      \renewcommand{\rmdefault}{ptm}
      \renewcommand{\ttdefault}{pcr}
      \smallcover\sffamily
      \begin{minipage}{0.47\textwidth}
         \begin{flushleft}
            \selectlanguage{french}
            \textbf{R\'esum\'e}\par
   }
   {
         \end{flushleft}
      \end{minipage}
   \end{lrbox}
   \restoregeometry
   \global\setbox\@resumeBoxGlobal=\vbox{\copy\@resumeBox}
   }

%%%%%%%%%%%%%%%%%%%%%%%%
% R'esum'e en anglais
%

\newsavebox{\@abstractBox}
\newbox{\@abstractBoxGlobal}

\newenvironment{abstract_}
   {
   \newgeometry{margin=1.5cm}
   \begin{lrbox}{\@abstractBox}%
      \renewcommand{\sfdefault}{phv}
      \renewcommand{\rmdefault}{ptm}
      \renewcommand{\ttdefault}{pcr}
      \smallcover\sffamily
      \begin{minipage}{0.47\textwidth}
         \begin{flushleft}
            \selectlanguage{english}
            \textbf{Abstract}\par
   }
   {
         \end{flushleft}
      \end{minipage}
   \end{lrbox}
   \restoregeometry
   \global\setbox\@abstractBoxGlobal=\vbox{\copy\@abstractBox}
   }

%%%%%%%%%%%%%%%%%%%%%%%%%%%%
% Mots cl'es en franc,ais
%

\newsavebox{\@keywordsBox}
\newbox{\@keywordsBoxGlobal}

\newenvironment{keywords}
   {
   \newgeometry{margin=1.5cm}
   \begin{lrbox}{\@keywordsBox}%
      \renewcommand{\sfdefault}{phv}
      \renewcommand{\rmdefault}{ptm}
      \renewcommand{\ttdefault}{pcr}
      \smallcover\sffamily
      \begin{minipage}{0.47\textwidth}
         \begin{flushleft}
            \selectlanguage{english}
            \textbf{Key Words}\par
   }
   {
         \end{flushleft}
      \end{minipage}
   \end{lrbox}
   \restoregeometry
   \global\setbox\@keywordsBoxGlobal=\vbox{\copy\@keywordsBox}
   }

%%%%%%%%%%%%%%%%%%%%%%%%%
% Mots cl'es en anglais
%

\newsavebox{\@motsclesBox}
\newbox{\@motsclesBoxGlobal}

\newenvironment{motscles}
   {
   \newgeometry{margin=1.5cm}
   \begin{lrbox}{\@motsclesBox}%
      \renewcommand{\sfdefault}{phv}
      \renewcommand{\rmdefault}{ptm}
      \renewcommand{\ttdefault}{pcr}
      \smallcover\sffamily
      \begin{minipage}{0.47\textwidth}
         \begin{flushleft}
            \selectlanguage{french}
            \textbf{Mots cl\'es}\par
   }
   {
         \end{flushleft}
      \end{minipage}
   \end{lrbox}
   \restoregeometry
   \global\setbox\@motsclesBoxGlobal=\vbox{\copy\@motsclesBox}
   }

\global\newcount\@startingPage

\newcommand{\makecover}{%
   \singlespacing
   \cleardoublepage % go to odd page
   \thispagestyle{empty} % no numbering
   \phantom{placeholder} % doesn't appear on page
   \newpage % on even page
   \newgeometry{margin=1.5cm}
   \thispagestyle{empty} % no numbering
   \@startingPage\thepage\relax
      {
      \@drawbackground
      \vspace{-5.5\baselineskip}
      \renewcommand{\sfdefault}{phv}
      \renewcommand{\rmdefault}{ptm}
      \renewcommand{\ttdefault}{pcr}
      \sffamily
      \fontsize{12pt}{14pt}
      \selectfont
      \begin{quote}
         \begin{flushleft}
            \textbf{\Largecover\textcolor{@textgray}{Kevin \textsc{Quirin}}}
         \end{flushleft}

         \begin{flushleft}
            \textbf{\textcolor{@textgray}{Faisceautisation de
                Lawvere-Tierney en théorie des types homotopique}}
         \end{flushleft}
         \begin{flushleft}
            \textbf{\textcolor{@textgray}{Lawvere-Tierney
                sheafification in Homotopy Type Theory}}
         \end{flushleft}
      \end{quote}

      \vfill

      \noindent
      \hfill\parbox[t][][b]
            {0.47\textwidth}
            {\makebox[0.47\textwidth]{\usebox{\@resumeBoxGlobal}}}
      \hfill
      \hfill\parbox[t][][b]
            {0.47\textwidth}
            {\makebox[0.47\textwidth]{\usebox{\@abstractBoxGlobal}}}
      \hfill

      \vfill

      \noindent
      \hfill\parbox[t][][b]
            {0.47\textwidth}
            {\makebox[0.47\textwidth]{\usebox{\@motsclesBoxGlobal}}}
      \hfill
      \hfill\parbox[t][][b]
            {0.47\textwidth}
            {\makebox[0.47\textwidth]{\usebox{\@keywordsBoxGlobal}}}
      \hfill

      \vfill

      \hspace{0.45\textwidth}\raisebox{-2.2\baselineskip}
                                      [0pt][0pt]
                                      {\colorbox{@bandeaucolor}
                                                {\parbox[s]
                                                        [2\baselineskip]
                                                        [c]
                                                        {0.55\textwidth}
                                                        {~\hfill
                                                         \textcolor{white}
                                                                   {L'UNIVERSIT\'E NANTES ANGERS LE MANS}
                                                         \hfill~}}}%
      }%
   \restoregeometry%
} % \makecover
\makeatother

\AtEndDocument{\makecover}


\begin{document}

\begin{resume}
  Le but principal de cette thèse est de définir une extension de la
  traduction de double-négation de Gödel à tous les types tronqués,
  dans le contexte de la théorie des types homotopique.
  Ce but utilisera des théories déjà existantes, comme la théorie des
  faisceaux de Lawvere-Tierney, que nous adapterons à la théorie des
  types homotopiques. En particulier, on définira le fonction de
  faisceautisation de Lawvere-Tierney, qui est le principal théorème
  présenté dans cette thèse.

  Pour le définir, nous aurons besoin de concepts soit déjà définis
  en théorie des types, soit non existants pour l'instant. En
  particulier, on définira une théorie des colimits sur des graphes,
  ainsi que leur version tronquée, et une notion de modalités
  tronquées basée sur la définition existante de modalité.

  Presque tous les résultats présentés dans cette thèse sont
  formalisée avec l'assistant de preuve Coq, muni de la
  librairie~\cite{hottlib}.
\end{resume}

\begin{abstract_}
  The main goal of this thesis is to define an extension of Gödel
  not-not translation to all truncated types, in the setting of
  homotopy type theory. 
  This goal will use some existing theories, like Lawvere-Tierney
  sheaves theory in toposes, we will adapt in the setting of homotopy
  type theory. In particular, we will define a Lawvere-Tierney
  sheafification functor, which is the main theorem presented in this
  thesis.

  To define it, we will need some concepts, either already defined in
  type theory, either not existing yet. In particular, we will define
  a theory of colimits over graphs as well as their truncated version,
  and the notion of truncated modalities, based on the existing
  definition of modalities.

  Almost all the result presented in this thesis are formalized with
  the proof assistant Coq together with the library~\cite{hottlib}.
\end{abstract_}

\begin{motscles}
  Théorie des types, homotopie, faisceautisation, Coq
\end{motscles}

\begin{keywords}
  Type theory, homotopy, sheafification, Coq
\end{keywords}


\newgeometry{margin=1.5cm}
\begin{titlingpage}
  {
    \singlespacing
    \makeatletter
    \@drawbackground
    \renewcommand{\sfdefault}{phv}
    \renewcommand{\rmdefault}{ptm}
    \renewcommand{\ttdefault}{pcr}
    \fontsize{12pt}{14pt}
    \sffamily
    \selectfont
    
    \vspace{-3.75\baselineskip}
    \hspace{0.5\textwidth}
    \raisebox{0pt}
    [0pt]
    [2\baselineskip]
    {\parbox[t][][c]
      {0.5\textwidth}
      {\raggedright\textcolor{@doctoratcolor}{\begin{spacing}{0.75}\Hugecover{Kevin \textsc{Quirin}}\end{spacing}}}}
    
    \smallcover
    \vspace{\baselineskip}
    
    \noindent\textsl{M\'emoire pr\'esent\'e en vue de l'obtention du}\par
    \noindent\textsl{\textbf{grade de Docteur
        de \parbox[t]{0.79\textwidth}{l'École nationale supérieure
          des Mines de Nantes}}}\par
    \noindent\textsl{\textbf{\phantom{grade de }Label européen}}\par
    \noindent\textsl{sous le label de l'Universit\'e de Nantes Angers Le Mans}\par
    
    \vspace{\baselineskip}
    
    \footnotesizecover
    \noindent\textbf{\'Ecole doctorale : Sciences et technologies de
      l'information, et mathématiques}\par
    
    \vspace{\baselineskip}
    
    \footnotesizecover
    \noindent\textbf{Discipline : Informatique et applications,
      section CNU 27}\par
    \noindent\textbf{Unit\'e de recherche
      : \parbox[t]{0.8\textwidth}{Laboratoire d'informatique de
        Nantes-Atlantique (LINA)}}\par
    
    \vspace{\baselineskip}
    
    \footnotesizecover
    \noindent\textbf{Soutenue le FIXME}\par
    \noindent\textbf{Th\`ese n\textdegree\ : FIXME}\par
    
    \vfill
    
    \hfill
    \parbox{0.75\textwidth}
    {\begin{flushright}
        \Largecover\textcolor{@titlecolor}{\textbf{Lawvere-Tierney
            sheafification in \\ Homotopy Type Theory}}\\
      \end{flushright}}
    
    \vfill
    \vfill
    \vfill
    
    \newsavebox{\@committeebox}
    \sbox{\@committeebox}{\@committee}
    \newlength{\committeeboxwidth}
    \settowidth{\committeeboxwidth}{\usebox{\@committeebox}}
    \begin{center}
      \textcolor{@titlecolor}{\textbf{\smallcover JURY}}\par
      \vspace{1em}
      {\centering\usebox{\@committeebox}}%
    \end{center}
  }
\end{titlingpage}%
\makeatother
\restoregeometry

\frontmatter%

\begin{abstract}
  The main goal of this thesis is to define an extension of Gödel
  not-not translation to all truncated types, in the setting of
  homotopy type theory. 
  This goal will use some existing theories, like Lawvere-Tierney
  sheaves theory in toposes, we will adapt in the setting of homotopy
  type theory. In particular, we will define a Lawvere-Tierney
  sheafification functor, which is the main theorem presented in this
  thesis.

  To define it, we will need some concepts, either already defined in
  type theory, either not existing yet. In particular, we will define
  a theory of colimits over graphs as well as their truncated version,
  and the notion of truncated modalities, based on the existing
  definition of modalities.

  Almost all the result presented in this thesis are formalized with
  the proof assistant Coq together with the library~\cite{hottlib}.

\end{abstract}

\begin{otherlanguage}{french}
  
\begin{abstract}
  Le but principal de cette thèse est de définir une extension de la
  traduction de double-négation de Gödel à tous les types tronqués,
  dans le contexte de la théorie des types homotopique.
  Ce but utilisera des théories déjà existantes, comme la théorie des
  faisceaux de Lawvere-Tierney, que nous adapterons à la théorie des
  types homotopiques. En particulier, on définira le fonction de
  faisceautisation de Lawvere-Tierney, qui est le principal théorème
  présenté dans cette thèse.

  Pour le définir, nous aurons besoin de concepts soit déjà définis
  en théorie des types, soit non existants pour l'instant. En
  particulier, on définira une théorie des colimits sur des graphes,
  ainsi que leur version tronquée, et une notion de modalités
  tronquées basée sur la définition existante de modalité.

  Presque tous les résultats présentés dans cette thèse sont
  formalisée avec l'assistant de preuve Coq, muni de la
  librairie~\cite{hottlib}.
\end{abstract}  

\end{otherlanguage}

\chapter{Acknowlegdments}

\todo[inline]{Write the acknowledgments}
%%% Local Variables:
%%% mode: latex
%%% TeX-master: "main"
%%% End:

\newpage
\tableofcontents

\mainmatter%
\begin{otherlanguage}{french}
  
\chapter{Résumé en français}

\section{Introduction}

La théorie des types homotopique est une branche nouvelle des
mathématiques et de l'informatique, exhibant un lien fort, mais
surprenant entre la théorie des $\omega$-catégories et la théorie de
types. Ce domaine se situe donc à la frontière entre les mathématiques
pures et l'informatique. Un des but de la recherche sur ce sujet est
d'utiliser la théorie des types homotopique comme une nouvelle
fondation des mathématiques, remplaçant par exemple la théorie des
ensembles de Zermelo-Fr\ae nkel. Ses liens forts avec la théorie des
types donneraient aux mathématiciens la possibilité de formaliser
leurs travaux avec un assistant de preuve comme
Coq~\cite{coq:refman:8.4}, Agda~\cite{norell2007towards} ou
Lean~\cite{lean}. En effet, les erreurs dans les articles de recherche
en mathématiques semblent inévitables~\cite{vv-univ-f}, et une preuve
vérifiée par ordinateur peut inspirer plus confiance qu'une preuve
vérifiée par un humain. Les exemples les plus célèbres de preuves
vérifiées par ordinateur sont le {\em théorème des quatre couleurs}
(pour colorier une carte telle qu'aucuns pays voisins aient la même
couleur, il suffit de quatre couleurs) par Gonthier et
Werner~\cite{gonthier-four-color} avec Coq, le {\em théorème de
Feit-Thomson} (tout groupe fini d'ordre impair est résoluble) par
Gonthier et al.~\cite{gonthier-feit} avec Coq, la preuve originelle du
{\em théorème de Jordan} (toute courbe simple continue fermée divise le plan
en une partie bornée ``intérieure'' et une partie non bornée
``extérieure'') par Hales~\cite{hales-jordan} avec Mizar, ou la {\em
  conjecture de Kepler} (les façons les plus compactes d'empiler des
sphères sont les empilements clos cubiques et hexagonaux) par Hales et
al.~\cite{hales-kepler} avec Isabelle et HOL Light.

Un des avantages de la théorie des types par rapport à la théorie des
ensembles est la propriété de calculabilité de la théorie des types :
tout terme est identifié avec sa forme normale. Ainsi, un assistant de
preuve permet à l'utilisateur de simplifier automatiquement toutes les
expressions, alors qu'une preuve sur papier requiert de faire toutes
ces calculs à la main. La théorie des ensembles ne partage pas cette
propriété de calculabilité, et n'est donc pas aussi pratique à
utiliser comme base théorique pour un assistant de preuve. Cependant,
cette propriété de calculabilité nous empêche d'utiliser des
propriétés classiques, telles que le tiers exclu ; en général, il est
impossible de prouver qu'une proposition est soit vraie, soit fausse.

L'ingrédient principal de la théorie des types homotopique, faisant le
lien entre les mathématiques et l'informatique, est l'isomorphisme de
Curry-Howard: on peut indifféremment parler de preuve ou de programme,
les deux décrivent les mêmes objets {\em via} une correspondance. Par
exemple, en théorie de types, la séquence de symboles $A\to B$ peut
être vue comme le type des programmes prenant un argument de type $A$
et produisant une sortie de type $B$ ou comme le type des preuves que
$A$ implique $B$. Une chose qu'implique cette correspondance est qu'il
peut exister plusieurs preuves de ``$A$ implique $B$'', puisqu'il peut
y avoir plusieurs manières de produire une sortie de type $B$ à partir
d'une entrée de type $A$. Cette propriété est appelée {\em
  proof-relevance}, alors que ZFC est considérée comme {\em
  proof-irrelevant}: si un lemme a été prouvé, la façon de le prouver
peut être oublié, elle n'a aucune importance.

À part le manque de propriétés classique, un problème de la théorie
des types est la notion d'égalité. On a deux possibilités:
\begin{itemize}
\item une égalité intentionnelle, ou définitionnelle ; deux objets
  sont égaux s'ils sont définis de la même manière, \ie{} si l'un peut
  être échangé avec l'autre sans changer le sens. Par exemple, le
  nombre naturel $1$ et le successeur du nombre naturel 0 sont
  intentionnellement égaux. En théorie des types, on ajoute des règles
  à cette égalité comme la règle $\beta$ ($(\lambda\ x,\ f\ x)y = f\
  y$) and $\eta$.
\item une égalité extensionnelle, ou propositionelle ; deux objets
  sont considérés égaux s'ils se comportent de la même façon. Par
  exemple, étant donnés deux nombres naturels $a$ and $b$, $a+b$ et
  $b+a$ sont extensionnellement mais pas intentionnellement égaux ; on
  a besoin de le prouver.
\end{itemize}
En théorie des ensembles, on utilise traditionnellement une égalité
extensionnelle, en affirmant que deux ensembles sont égaux si et
seulement s'ils ont les mêmes éléments. En théorie des types,
l'égalité intentionnelle est une notion meta-théorique ; seul le
type-checker (comme Coq) peut y accéder. On ne peut pas l'exprimer
dans la théorie elle-même car on sait que la théorie des types
extensionnelle est indécidable~\cite{hofmann1995extensional} (étant
donnés un terme $p$ et un type $P$, il peut être indécidable de
vérifier que $p$ est bien une preuve de $P$). L'égalité
propositionnelle est un concept interne à la théorie, définie comme un
type inductif
inductive type
\[ \mathrm{Id} (A:\Type) (a:A) : A \to \Type
\] généré par un seul constructeur
\[ \idpath : \mathrm{Id}_A(a,a),
\] 
et le type $\mathrm{Id}(A,a,b)$ sera noté $a=_A b$ ou $a=b$. Ce type
identité n'est en fait pas satisfaisant. L'idée de Martin-Löf était de
copier l'égalité mathématique, mais le type identité n'y arrive
pas. En effet, un problème avec cette égalité est que les types $a=b$
peuvent être habités de différents manières -- au moins, il n'est pas
prouvable que pour tous $p,q:a=b$, $p=q$ ; cette propriété, appelée
UIP (unicité des preuves d'égalité) a été prouvée
dans~\cite{Hofmann96thegroupoid} être indépendante de la théorie des
types intentionnelle. Un autre problème est que cette égalité est
définie pour tous les types, et ne se comporte pas bien vis-à-vis de
certains constructeurs de types ; par exemple, l'extensionalité
fonctionnelle, affirmant que deux fonctions sont égales dès qu'elles
sont égales point-à-point, ne peut pas être montrée.

Cependant, il ne faut pas jeter les types identité. Vers 2006,
Vladimir Voevodsky, et Steve Awodey et Michal
Warren~\cite{awodey-warren} ont donné indépendamment une nouvelle
interprétation des types identité : les types sont maintenant vus
comme des espaces topologiques, les habitants des types comme des
points, et un élément $p:\mathrm{Id}(A,a,b)$ peut être lu comme
\begin{quotation}
  Dans l'espace $A$, $p$ est une homotopie (ou un chemin continu)
  entre les points $a$ et $b$.
\end{quotation}
Avec cette interprétation, il paraît normal de ne pas satisfaire UIP :
il peut y avoir plusieurs (\ie{} non homotopiques) chemins entre deux
points (on peut penser à un beignet). Le second problème a été résolu
vers 2009, quand Vladimir Voevodsky a énoncé l'axiome d'univalence:
deux types sont égaux exactement quand ils sont isomorphes. De façon
surprenante, cette axiome implique la compatibilité des types identité
avec certains constructeurs de types: il implique l'extensionalité
fonctionnelle, et il semble qu'il implique aussi que le type identité
sur les {\em streams} coïncide avec la
bisimulation~\cite{licata14uafs}.

Le projet originel de Voevodsky~\cite{vv-nsf} était de donner un outil
aux mathématiciens pour qu'il puissent vérifier leurs preuves sur
ordinateur. La théorie des types homotopique semble être une bonne
base pour ça, mais il manque le {\em tiers-exclu}, un des principes
préférés des mathématiciens :
\begin{quotation}
  Toute proposition est soit vraie, soit fausse.
\end{quotation}
Le but principal de cette thèse est d'ajouter ce principe à la théorie
des types homotopique, sans perdre certaines propriétés (décidabilité,
canonicité, constructivisme).

Commençons par remarquer qu'on sait déjà transformer une logique
intuitionniste en une logique classique, avec la traduction de
Gödel-Gentzen définie dans la figure~\ref{fig:GG-trans-fr}
\begin{figure}[ht]
  \centering

  \[x^N \defeq \lnot\lnot x\text{ quand $x$ est atomique}\]
  \[(\phi \land \psi)^N \defeq \phi^N \land \psi^N \qquad
  (\phi\lor \psi)^N \defeq \lnot (\lnot\phi^N \land \lnot \psi^N)\]
  \[(\phi \to \psi)^N \defeq \phi^N \to \psi^N \qquad
    (\lnot \psi)^N \defeq \lnot \phi^N\]
  \[(\forall\ x,\ \phi)^N \defeq \forall\ x,\ \psi^N \qquad
  (\exists\ x,\ \phi)^N \defeq \lnot\forall\ x\,\ \lnot\phi^N\]
  \caption{Traduction de Gödel-Gentzen}
  \label{fig:GG-trans-fr}
\end{figure}

Un théorème de correction affirme qu'une formule $\phi$ est
classiquement prouvable si et seulement si $\phi^N$ est
intuiitionnistiquement prouvable. Bien que cette traduction ne
fonctionne qu'avec la logique, la même idée peut être utilisée pour
toute la théorie des types, comme dans
\cite{jaber2012extending,forcing2016}.
L'idée derrière une traduction est, en partant d'une théorie source
(compliquée) $\mathcal S$, de traduire tous les termes $t$ de
$\mathcal S$ en des termes $[t]$ d'une théorie source $\mathcal T$,
supposée sue consistante. La propriété fondamentale d'une traduction
est sa correction, ie{} si on peut prouver un théorème de correction
affirmant que si un terme $x$ est de type $X$ dans $\mathcal S$, alors
la traduction $[x]$ de $x$ est de type $\Lbrack X \Rbrack$, où
$\Lbrack\cdot\Rbrack$ est la traduction des types. Cette propriété,
avec une preuve que $\Lbrack\zero\Rbrack$ n'est pas habité dans
$\mathcal T$, assure que la théorie $\mathcal S$ est consistante. On
peut dire qu'une traduction correcte est un moyen de donner un nom
dans la théorie cible $\mathcal T$ aux objets de $\mathcal S$ inconnus
de $\mathcal T$.

Les théoriciens de ensembles peuvent remarquer que c'est très proche
de la méthode de {\em forcing}, inventée en 1962 par Paul
Cohen~\cite{cohen1966}. Son application historique, et la plus connue,
la preuve d'indépendance de la négation de l'hypothèse du continu avec
ZFC. De la même façon que les traductions, le forcing ne peut montrer
que des résultats de consistance relative, \eg{} ZFC+$\lnot$HC est
consistant si ZFC est consistent. Cette méthode est aujourd'hui un
ingrédient clef des théoriciens de ensembles.

Nous savons déjà qu'adapter des résultats de théorie des ensembles à
la théorie des types n'est pas facile, ces deux théories étant très
différents. La théorie des types est plus proche de la théorie des
topoi, et la théorie des types homotopique encore plus proche de la
théorie des topoi supérieurs. Heureusement, Myles Tierney a donné en
1972 un équivalent du forcing en théorie des topoi~\cite{tierney1972},
en utilisant la notion de faisceaux. À l'origine, les faisceaux
n'existaient que sur des topoi de préfaisceaux (les topoi de foncteurs
d'une catégorie $\mathbf C$ vers la catégorie $\mathbf{Sets}$). Ces
faisceaux sont appelés faisceaux de Grothendieck, et correspondent aux
objets $F$ tels que toute fonction $X \to F$ peut être définie de
façon équivalence soit sur $X$ tout entier, soit sur tous les objets
d'un recouvrement ouvert de $X$. Ce concept a été étendu par William
Lawvere et Myles Tierney, autorisant les objets de n'importe quel
topos à être des faisceaux. Ils correspondent alors aux objets $F$
tels que toute fonction $X\to F$ peut être définie de façon
équivalente soit sur $X$ tout entier, soit sur un sous-objet dense de
$X$. L'existence d'un {\em foncteur de faisceautisation} de n'importe
quel topos $\mathcal T$ vers son topos de faisceau $\Sh{}(\mathcal
T)$, adjoint à gauche de l'inclusion, permet de construire une topos
satisfaisant plus de propriétés que le topos de départ. Les faisceaux
proviennent d'une topologie fixée au départ, qui peut être vue comme
un opérateur sur la ``logique'' du topos (le classifiant des
sous-objets), idempotent, préservant $\top$ et commutant avec les
produits. La double négation est en fait une topologie sur n'importe
quel topos $\mathcal T$, et alors le topos $\Sh{\lnot\lnot}(\mathcal
T)$ satisfait de bonnes propriétés il est booléen (\ie{} satisfait le
tiers exclu), et la négation de l'hypothèse du continu est vraie dans
ce topos~\cite{maclanemoerdijk}.

En théorie des topoi supérieurs, ce $(\infty,1)$-foncteur de
faisceautisation n'a été défini dans~\cite{lurie} que dans le cas des
faisceaux de Grothendieck, laissant la théorie de Lawvere-Tierney
inexplorée. Il y a donc un défi double dans notre quête de tiers-exclu
en théorie des types homotopique : le premier est de formaliser le
résultat de la théorie des topoi, et le second est de l'étendre au cas
de la théorie des types homotopique, en utilisant des résultats de la
théorie des $(\infty,1)$-topoi.

\paragraph*{But de la thèse}
Le but principal de cette thèse est de donner une définition du
foncteur de faisceautisation de Lawvere-Tierney dans le contexte de la
théorie des types homotopique. Pour cela, on aura d'abord besoin d'une
théorie des colimites en théorie des types homotopique. Puis, comme
notre définition de la faisceautisation sera faite inductivement sur
le niveau de troncation, on définira une version tronquée ce ces
colimites, et une version tronquée des modalités exactes à gauche. 
Presque tous ces résultats sont vérifiés par ordinateur par l'assistant de
preuve Coq ; la plupart d'entre eux sont disponibles sur mon compte
Github \url{https://github.com/KevinQuirin}.

Notre étude approfondie des modalités nous a aussi amené à définir une
traduction de théorie des types associée à une modalité exacte à
gauche, et à écrire une extension pour Coq pour manipuler
automatiquement cette traduction.

\section{Théorie des types homotopique}
Dans cette section, nous allons décrire brièvement le cadre dans
lequel on se place, la théorie des types homotopique. La définition
compacte de cette théorie pourrait être
\[
  \mathrm{HoTT} = \mathrm{MLTT}+\mathrm{UA}+\mathrm{HIT}
\]
où MLTT est la théorie des types de Martin-Löf (ou théorie des types
dépendants), UA est l'axiome
d'univalence et HIT est l'abréviation des types inductifs supérieurs.
Pour une description plus complète de MLTT, le lecteur pourra se référer
à~\cite{hofmann1997syntax} ou \cite{hottbook}.

Dans la théorie des ensembles de Zermelo-Frank\ae l, la formule
logique la plus basique est
\[ x\in E\]
où $x$ et $E$ sont des ensembles. En théorie des types dépendants, un
jugement similaire serait 
\[ a : A\]
qui doit être lu comme ``$a$ est de type $A$''. La différence
principale avec la relation d'appartenance est qu'un élément $a$
n'appartient qu'à un et un seul type, alors qu'il est possible
d'écrire $x\in E$ et $x\in F$ en théorie des ensembles (c'est la
définition de $x\in E\cap F$).

La théorie des types dépendants est basée sur la correspondance de
Curry-Howard, ou principe de ``proposition comme des types''. En
effet, on ne fait pas de différence entre les proposition et les
types ; $a:A$ sera lu indifféremment ``$a$ est de type $A$'' si $A$
est vu comme un type ou ``$a$ est une preuve de $A$'' quand $A$ est vu
comme une proposition. 

On ne rappelle pas dans ce résumé les règles d'introduction et
d'élimination des types, mais on donne le tableau suivant rappelant
les différents types existants:

\renewcommand{\arraystretch}{2}
\begin{tabular}{|l|l|l|}
  \hline
  Nom & Notation & Propositions comme des types \\
  \hline\hline
  Zéro & $\displaystyle{\zero}$ & $\displaystyle{\bot}$ \\
  \hline
  Un & $\displaystyle{\one}$ & $\displaystyle{\top}$ \\
  \hline
  Coproduit, sum & $\displaystyle{A+B}$ & $\displaystyle{A\lor B}$ \\
  \hline
  Fonction & $\displaystyle{A\to B}$ & $\displaystyle{A\To B}$ \\
  \hline
  Fonction dépendante, type Pi & $\displaystyle{\prodD x A {B\, x}}$ & $\displaystyle{\forall x,\,
                                                       B\, x}$ \\
  \hline
  Produit & $\displaystyle{A\times B}$ & $\displaystyle{A\land B}$ \\
  \hline
  Somme dépendante, type Sigma & $\displaystyle{\sumD x A {B\, x}}$ & $\displaystyle{\exists x,\, B\,
                                                    x}$ \\
  \hline
  Égalité, identité, chemin & $\displaystyle{a =_A b}$ & $\displaystyle{a = b}$
  \\
  \hline
\end{tabular}
\renewcommand{\arraystretch}{1}

Revenons simplement sur les types identité. Ils sont très utiles pour
affirmer l'égalité propositionnelle entre des objets. On note qu'ils
ne caractérisent pas l'égalité jugementale, qu'on considère comme
appartenant à la méta-théorie. Les types identités donnent à tout type
une structure d'$\omega$-groupoïde, \ie{} la structure d'une
$\omega$-catégorie où toutes les flèches sont inversibles. En
particulier, ils vérifient la réflexion, la transitivité, la symétrie,
l'associativité, le pentagone de Maclane, \etc{}

Les types identités se comportent bien vis-à-vis des fonctions. Si
$f:A\to B$ est une fonction, alors pour tous $x,y:A$, il existe une
fonction 
\[\ap f: x=y \to f(x) = f(y)\]
compatible avec la structure d'$\omega$-groupoïde. 

Avant de lier les
type identités avec les types d'équivalence, définissons d'abord les
équivalences.
\begin{defifr}
  Soient $A,B:\Type$ et $f:A\to B$. On dit que $f$ est une
  équivalence, noté $\IsEquiv(f)$ s'il existe
  \begin{itemize}
  \item une fonction $g:B\to A$, appelée inverse de $f$
  \item un terme $\displaystyle{\retr_f: \prodD x A {g(f(x)) = x}}$,
    appelé la rétraction de l'équivalence
  \item un terme $\displaystyle{\sect_f: \prodD x A {f(g(x)) = x}}$,
    appelé la section de l'équivalence
  \item un terme $\displaystyle{\adj_f: \prodD x A {\ap f {\retr_f x} =
        \sect_f(f\, x)}}$, appelé l'adjonction de l'équivalence.
  \end{itemize}
  On dit que $A$ et $B$ sont équivalents, noté $A\simeq B$ s'il existe
  $f:A\to B$ qui soit une équivalence. 
\end{defifr} 

On note que si $A=B$, alors il existe une équivalence canonique entre
$A$ et $B$, et on note $\mathrm{idtoequiv}:A = B \to A \simeq B$.
On peut alors énoncer l'axiome d'univalence:
\begin{axfr}
  Pour tous types $A$ et $B$, la fonction
  \[ \mathrm{idtoequiv} : A = B \to A\simeq B \]
  est une équivalence.
\end{axfr}
 
Avec cet axiome, il semble que les types identité sont compatibles
avec tous les constructeurs de types: 
\begin{itemize}
\item l'égalité dans les produits correspond à l'égalité des
  composantes
\item l'égalité dans les sommes dépendantes correspond à l'égalité
  des composantes, la deuxième transportée par la première
\item l'égalité dans les fonctions (dépendantes ou non) correspond à
  l'égalité point à point
\end{itemize}

Pour obtenir la théorie des types homotopiques, il reste à rajouter un
moyen de construire des types dont on contrôle -- en partie -- les
espaces de chemins itérés~: les types inductifs supérieurs (HIT). Les
HIT fonctionnent de la même manière que les types inductifs de Coq,
mais on peut aussi rajouter des constructeurs pour les espaces de
chemins.

\begin{exmfr}
  Le premier exemple est le cercle $\Sone$. Il est généré par les
  constructeurs
  \[ \left| 
      \begin{array}{lll}
        \baseS & : & \Sone \\
        \loopS & : & \baseS = \baseS
      \end{array}
    \right.
  \]
  On peut le représenter par
  \[
    \xymatrix@R=0.1em{ \bullet \ar@(ur,ul)[]_\loopS \\ \baseS}
  \]
\end{exmfr}

Comme pour les types inductifs, on donne aussi des éliminateurs pour
les HIT.

\begin{exmfr}
  Donnons l'éliminateur non-dépendant de $\Sone$. Si $P$ est un type,
  $b:P$ et $p:b=b$, alors il existe une fonction 
  \[\Sone_{\mathrm{rec}} : \Sone \to P\]
  tel que $\Sone_{\mathrm{rec}}(\baseS) \equiv b$ et
  $\ap{\Sone_{\mathrm{rec}}}(\loopS) = p$.

  L'éliminateur dépendant est un peu plus compliqué. Si
  $P:\Sone\to\Type$ est une famille de type sur $\Sone$, $b:(P\baseS)$
  et $p:\transport_P^\loopS(b) = b$, alors il existe un terme
  \[ \Sone_{\mathrm{ind}} : \prodD x \Sone {P\, x}\]
  tel que $\Sone_{\mathrm{ind}}(base) \equiv b$ et
  $\ap{\Sone_{\mathrm{ind}}}(\loopS) = p$.
\end{exmfr}

\section{Colimites}

On voit dans~\cite{lumsdaine} que la présence des sommes dépendantes
en théorie des types permet de construire des limites sur des
graphes. Cependant, alors que ce n'est qu'une notion duale, on ne sait
pas traiter le cas des colimites dans MLTT.  On va voir dans cette
partie que les types inductifs supérieurs permettent de gérer les
colimites au-dessus de graphes, et même au-dessus de catégories dans
certains cas.

Commençons par rappeler les définitions de graphes et diagrammes.


\begin{defifr}[Graphe]
  Un {\em graphe} $G$ est la donnée de
  \begin{itemize}
  \item un type $G_0$ de sommets ;
  \item pour tous $i,j:G_0$, un type $G_1(i,j)$ d'arêtes.
  \end{itemize}
\end{defifr}

\begin{defifr}[Diagramme]
  Un {\em diagramme} $D$ sur un graphe $G$ est la donnée de
  \begin{itemize}
  \item pour tous $i:G_0$, un type $D_0(i)$ ;
  \item pour tous $i,j:G_0$ et tous $\phi : G_1(i,j)$, une flèche $D_1(\phi)
    : D_0(i) \to D_0(j)$
  \end{itemize}
\end{defifr}

Comme dans le contexte catégorique, on va essayer de définir une
colimite comme un type qui forme un cocone sur le graphe désiré, de
façon universelle. Dans la suite, on se fixe un graphe $G$ et un
diagramme $D$ au-dessus de $G$.

\begin{defifr}
  Soit $Q$ un type. Un cocone sur $D$ dans $Q$ est la donnée de
  flèches $q_i : D_i \to Q$, et pour tous $i,j:G$ et $g:G(i,j)$, une
  homotopie $q_j \circ D(g) \homot q_i$.
\end{defifr}

On peut postcomposer les cocones par des flèches. Plus précisément, si
$Q$ et $Q'$ sont des types, $f:Q \to Q'$ et $C$ un cocone sur $D$ dans
$Q$, alors il existe un cocone sur $D$ dans $Q'$ qui consiste à
postcomposer toutes les flèches de $C$ par $f$. Cela donne une flèche
\newcommand{\postcomposecocone}{\mathrm{postcompose}_{\mathrm{cocone}}}
\[\postcomposecocone : \cocone_D(Q) \to (Q':\Type) \to (Q \to Q') \to
  \cocone_D(Q') \]
L'autre sens correspond à notre définition de colimite:

\begin{defifr}[Colimite]
  Un type $Q$ est une colimite de $D$ s'il existe un cocone $C$ sur
  $C$ dans $Q$, qui est universel, \ie{} tel que
  $\postcomposecocone(C,Q')$ soit une équivalence.
\end{defifr}

Les colimites sont compatibles avec différentes opérations sur les
diagrammes, que nous n'expliciterons pas ici. On notera simplement que
deux diagrammes équivalents ont des colimites équivalentes, donnant
par suite l'unicité à équivalence près des colimites.

Regardons un cas particulier de colimites qui nous intéressera plus
tard~: les constructions que Van Doorn et de Boulier.
Elles généralisent le théorème de théorie des topoi affirmant que tout
épimorphisme est la colimite de sa kernel pair. En théorie des topoi
supérieurs, le résultat reste vrai en remplaçant ``kernel pair'' par
``nerf de \v{C}ech''. Cependant, on ne sait pas définir en
théorie des types les nerf de
\v{C}ech, qui sont une version particulière d'objets simpliciaux. On
va donc donner un autre diagramme dépendant d'une fonction dont la
colimite est l'image de la fonction. On commence par la construction
de Van Doorn, qui correspond à une fonction $A\to\one$, et donnant la
$(-1)$-troncation comme colimite d'un diagramme.



\begin{prop}[Construction de Van Doorn]

  Soit $A:\Type$. On définit le type inductif supérieur $TA$ comme la
  colimite (le coégaliseur) de
  \[ \xymatrix{ A\times A \ar@<-.5ex>[r]_-{\pi_2}
      \ar@<.5ex>[r]^-{\pi_1} & A }.\]
  Alors la colimite de
  \[ \xymatrix{
      A \ar[r]^-q& TA \ar[r]^-q& TTA \ar[r]^-q& TTTA \ar[r]^-q& \dots
    }\]
  est $\|A\|_{-1}$.
\end{prop}

Maintenant, soient $A,B:\Type$ et $f:A\to B$. On définit la kernel pair de $f$
comme la colimite de \[ \xymatrix{ A\times_B A \ar@<-.5ex>[r]_-{\pi_2}
      \ar@<.5ex>[r]^-{\pi_1} & A }.\]
en d'autres termes, $\KP(f)$ est le type inductif supérieur généré par
\[\left|
    \begin{array}{lll}
      \kp &:& A\to \KP(f) \\
      \alpha &:& \displaystyle{\prodD {x,y}{A}{f\, x = f\, y \to kp\,x=\kp\,y}}
    \end{array}
  \right.\]
En utilisant l'éliminateur des colimites, on peut construire une
fonction $\widehat
f:\KP(f) \to B$, telle que le digramme suivant commute
\[\xymatrix{
    A \ar[r]^-{\kp} \ar[rd]_-f & \KP(f) \ar[d]^-{\widehat f} \\
    &B
}\]
On peut alors construire
$KP(\widehat f)$ et une fonction $\widehat{\widehat
  f}:\KP(\widehat f) \to B$, \etc{}
On a le résultat suivant
\begin{prop}[Construction de Boulier]
  Pour tout $f:A\to B$, $\im(f)$ est la colimite de la kernel pair
  itérée de $f$
\[\xymatrix{
  A \ar[r] & \KP(f) \ar[r] & \KP\left(\widehat f\right) \ar[r]& \KP\left(\widehat{\widehat f}\right) \ar[r]& \cdots
}\]
En particulier, si $f$ est une surjection, la colimite de ce diagramme
est $B$.
\end{prop}

Un problème de ces deux constructions est qu'ils ne préservent pas le
niveau de troncation (resp. les plongements). Si $A$ est un $\HProp$,
$TA$ peut avoir un niveau de troncation élevé (resp. si $f$ est un
plongement, $\widehat f$ ne l'est pas forcément). Ce problème peut
être résolu en considérant des graphes avec compositions, et en
définissant les deux types suivants
\[
  TA~\left|
    \begin{array}{lll}
      q&:& A\to TA \\
      \alpha&:& \displaystyle{\prodD {x,y} A {q\, x = q\, y}} \\
      \alpha_1&:& \displaystyle{\prodD x A {\alpha(x,x) = 1}}
    \end{array}
  \right.
  \quad ; \quad
  \KP(f)~\left|
    \begin{array}{lll}
      \kp&:&A\to \KP(f) \\
      \alpha&:& \displaystyle{\prodD {x,y} A {f\, x = f\, y \to q\, x = q\, y}} \\
      \alpha_1&:& \displaystyle{\prodD x A {\alpha(x,x,1) = 1}}
    \end{array}    
  \right.
\]
Les deux résultats énoncés précédemment restent vrais. Cependant, les
niveaux d'homotopie plus élevés ne sont pas préservés.
Comme pour construire le foncteur de faisceautisation, nous travaillerons à
niveau de troncation fixé, il nous faut un moyen de tronquer les
colimites.

On appelera $n$-colimite d'un diagramme $D$ un type $Q$ qui forme un
cocone sur $D$ et qui est universel par rapport à tous les
$n$-types. On a alors le résultat
\begin{lemfr}
  Soit $D$ un diagramme et $Q$ une colimite de $D$. Alors $\|Q\|_n$
  est une $n$-colimite de $\|D\|_n$, le diagramme où on a $n$-tronqué
  tous les types de $D$.
\end{lemfr}

\section{Modalités}

Comme dit dans l'introduction, le but principal de notre travail est
de construire, à partir d'un modèle $\mathfrak M$ de la théorie des
types homotopique, un autre modèle $\mathfrak M'$ qui satisfait 
de nouveaux principes.
On va utiliser un analogue aux modèles internes~\cite{kunen} de la
théorie des ensembles, représentés ici par des modalités exactes à
gauche.



\todo[inline]{Résumé modalités}

\section{Faisceaux}

\todo[inline]{Résumé faisceaux}

\section{Conclusion}
Commençons par un résumé de la thèse. La théorie des types homotopique
est un nouveau domaine de recherche, et se compose de la théorie des
types de Martin-Löf où on voit les types identité comme des
homotopies, à laquelle on ajoute l'axiome d'univalence, liant les
équivalences et les égalités, et les types inductifs supérieurs,
permettant de construire des types avec des égalités non-triviales.
Il semble exister un lien fort entre cette théorie et celle des topoi
supérieurs. Plus précisément, il semble qu'on puisse voir la théorie
des types homotopique comme le langage interne des $(\infty,1)$-topoi.

En théorie des types homotopiques, les types sont classifiés par leur
niveau de troncation, représentant la complexité de ses espaces de
boucles itérés. En particulier, si $X$ est $(n+1)$-tronqué, alors tous
les $x=y$ avec $x,y:X$ sont $n$-tronqués. On utilise cette propriété
pour construire un opérateur sur tous les types tronqués par induction
sur le niveau de troncation.

L'opérateur qu'on veut construire est une modalité. Les modalités sont
une version généralisée des localisations, qui sont elles-mêmes un
moyen de caractériser de façon équivalence la notion de sous-topos ;
cette équivalence est toujours vraie en théorie des topos
supérieurs~\cite[Section 6.2.2]{lurie}. En théorie des types
homotopique, une modalité est un opérateur $\modal$ sur $\Type$, muni
d'unités $\eta:\prodD X \Type {X\to\modal X}$ satisfaisant de bonnes
propriétés. Elles peuvent simplement être vue comme des monades
idempotentes. En se basant sur l'équivalence entre la théorie des
types et les $(\infty,1)$-topoi, on peut conjecturer que ces modalités
-- en fait, les modalités accessibles et exacte à gauche -- induisent
des sous-théories de types réflexives. Dans cette thèse, on veut
décrire une théorie des types classique (\ie{} qui satisfait le
principe du tiers-exclu) comme uns sous-théorie de la théorie des
types homotopique, en utilisant une modalité. On sait déjà que c'est
possible en théorie des topoi: en prenant n'importe quel topos
$\mathcal T$, et la topologie de Lawvere-Tierney de la double
négation, on peut construire le topos $\Sh{\lnot\lnot}(\mathcal T)$,
qui est booléen.

L'idée principale de notre travail est de remarquer que les topologies
de Lawvere-Tierney sur un topos, qui sont des opérateurs sur le
classifiant des sous-objets, peuvent être vues en théorie des types
homotopique comme des modalités sur $\HProp$, la deuxième couche de la
stratification des types. De plus, le foncteur de faisceautisation
correspond à étendre cette modalité tronquée à $\HSet$, la couche
suivante. On a donc cru possible d'étendre à nouveau à la couche
suivante, \etc{} pour finalement donner une modalité sur tous les
types tronqués. En fait, {\em modulo} quelque changements dans
plusieurs preuves -- en particulier la preuve impliquant des {\em
  kernel pair} de flèches -- et quelque sophistications dans les
preuves, la méthode décrite dans le cadre des topoi peut être répétée
indéfiniment pour construire une modalité sur tous les niveaux de
notre stratification.

Malheureusement, la gestion actuelle des univers par Coq ne nous
permet pas de formaliser complètement ce résultat ; cependant, une
grosse partie est vérifiée par ordinateur. Certaines parties ne
peuvent être vérifiées qu'en utilisant l'option (inconsistante)
\code{type-in-type} de Coq, autorisant à avoir $\Type^i : \Type^i$,
mais la définition complète ne peut pas être vérifiée complètement. 

\end{otherlanguage}


%%% Local Variables:
%%% mode: latex
%%% TeX-master: "main"
%%% End:

\chapter{Introduction}
\label{chap:intro}

Here is the intro.
\todo[inline]{Write the intro}
\chapter{Homotopy type theory}
\label{chap:hott}

\section{Dependent type theory}
\label{sec:mltt}
$a:A$: judgement ``$a$ is of type
  $A$''\nomenclature{$a:A$}{Judgement ``$a$ is of type $A$''}

\subsection{Empty and Unit types}
\label{ssec:unit_empty}

The first two types we will see are the Empty type (denoted
$\zero$\nomenclature{$\zero$}{Empty type}) and the Unit type (denoted
$\one$\nomenclature{$\one$}{Unit type}).
These are respectively the types with zero and one elements (named $\unittt$). Those two
types are dual to each other:
\begin{itemize}
\item having a term of type $\zero$ in the context allows to prove
  anything, while having a term of type $\one$ in the context is
  useless
\item dually, giving a term of type $\one$ is trivial, while giving a
  term of type $\zero$ is impossible (if the theory if consistent).
\end{itemize}

\begin{center}
  \AxiomC{$\Gamma\vdash x:\zero$}
  \AxiomC{$X:\Type$}
  \RightLabel{$\zero$-\textsc{elim}}
  \BinaryInfC{$\Gamma \vdash X$}
  \DisplayProof
  \qquad
  \AxiomC{}
  \RightLabel{$\one$-\textsc{intro}}
  \UnaryInfC{$\Gamma\vdash \unittt:\one$}
  \DisplayProof
\end{center}

Under the propositions-as-types principle, $\zero$ is the type always
false, and $\one$ the type always true. With a categorical point of
view, $\zero$ is an initial object and $\one$ is a terminal object.

\subsection{Coproduct}
\label{ssec:coproduct}
$A+B$: coproduct of $A$ and $B$\nomenclature{$A+B$}{Coproduct of
  types}

\begin{center}
  \AxiomC{$\Gamma\vdash a:A$}
  \RightLabel{$+$-$\textsc{intro}_L$}
  \UnaryInfC{$\Gamma\vdash \inl a : A+B$}
  \DisplayProof
  \qquad
  \AxiomC{$\Gamma\vdash b:B$}
  \RightLabel{$+$-$\textsc{intro}_R$}
  \UnaryInfC{$\Gamma\vdash \inr b : A+B$}
  \DisplayProof
  \vspace{1em}

  \AxiomC{$\Gamma\vdash p:A+B$}
  \AxiomC{$\Gamma,x:A\vdash c_A:C$}
  \AxiomC{$\Gamma,x:B\vdash c_B:C$}
  \RightLabel{$+$-\textsc{elim}}
  \TrinaryInfC{$\Gamma,\vdash\mathrm{sum\_rect}(p,c_A,c_B) : C$}
  \DisplayProof
\end{center}


\subsection{Dependent product}
\label{ssec:pi}
$A\to B$: types of arrows from $A$ to $B$\nomenclature{$A\to B$}{Type of arrows from $A$ to $B$}

$\prodD a A {B\, a}$ or $(x:A) \to B\, x$: dependent product over
$B$\nomenclature{$\prodD a A {B\, a}$}{Dependent product over $B$}

\begin{center}
  \AxiomC{$\Gamma,x:A\vdash b:B$}
  \RightLabel{$\prod$-\textsc{intro}}
  \UnaryInfC{$\Gamma\vdash \lambda\, (x:A),\, b : \prodD x A B$}
  \DisplayProof
  \qquad
  \AxiomC{$\Gamma\vdash f:\prodD x A B$}
  \AxiomC{$\Gamma\vdash a:A$}
  \RightLabel{$\prod$-\textsc{elim}}
  \BinaryInfC{$\Gamma\vdash f(a):B[a/x]$}
  \DisplayProof
\end{center}

\subsection{Dependent sum}
\label{ssec:sigma}
$A\times B$: product of $A$ and $B$\nomenclature{$A\times B$}{Product of types}


$\sumD a A {B\, a}$: dependent sum over $B$\nomenclature{$\sumD a A {B\, a}$}{Dependent sum over $B$}

\begin{center}
  \AxiomC{$\Gamma,x:A\vdash B:\Type$}
  \AxiomC{$\Gamma\vdash x:A$}
  \AxiomC{$\Gamma\vdash b:B[a/x]$}
  \RightLabel{$\sum$-\textsc{intro}}
  \TrinaryInfC{$\Gamma\vdash (a,b):\sumD x a B$}
  \DisplayProof
  \vspace{1em}

  \AxiomC{$\Gamma\vdash p:\sumD x A B$}
  \RightLabel{$\sum$-$\textsc{intro}_1$}
  \UnaryInfC{$\Gamma\vdash \pi_1 p: A$}
  \DisplayProof
  \qquad
  \AxiomC{$\Gamma\vdash p:\sumD x A B$}
  \RightLabel{$\sum$-$\textsc{intro}_2$}
  \UnaryInfC{$\Gamma\vdash \pi_2 p: B[\pi_1 p/x]$}
  \DisplayProof
\end{center}

\subsection{Inductive types}
\label{ssec:inductive}

\subsection{Paths type}
\label{ssec:path}
$a=_A b$ or $a=b$: type of paths from $a$ to $b$ in
$A$\nomenclature{$a=_A b$}{Type of paths from $a$ to $b$ in
  $A$}\nomenclature{$a=b$}{Type of paths from $a$ to $b$}

$1$ or $\idpath$ or $1_x$ or $\idpath_x$: constructor of
$x=x$\nomenclature{$1_x$ or $1$}{Constant path over
  $x$}\nomenclature{$\idpath_x$ or $\idpath$}{Constant path over $x$}

\begin{center}
  \AxiomC{$\Gamma\vdash a:A$}
  \RightLabel{$=$-\textsc{intro}}
  \UnaryInfC{$\Gamma\vdash \idpath_a : a =_A a$}
  \DisplayProof
  \vspace{1em}

  \AxiomC{$\Gamma\vdash P:A\to\Type$}
  \AxiomC{$\Gamma\vdash x,y:A$}
  \AxiomC{$\Gamma\vdash p:x=_A y$}
  \AxiomC{$\Gamma\vdash w:P\, x$}
  \RightLabel{$=$-\textsc{elim}}
  \QuaternaryInfC{$\Gamma\vdash \transport_P^p(x) : P\, y$}
  \DisplayProof
\end{center}

\section{Univalence axiom}
\label{sec:ua}

\section{Higher Inductives Types}
\label{sec:hit}

\subsection{Truncations}
\label{ssec:trunc}

$\|A\|_n$: $n$-truncation of $A$%
  \nomenclature{$\Vert\cdot\Vert$}{$n$-truncation of types}

\nomenclature{$\vert\cdot\vert_n$}{$n$-truncation of terms or arrows}

\section{Introduction to homotopy type theory}
\label{sec:hott}

\chapter{Higher modalities}
\label{chap:modalities}

As said in the introduction, the main purpose of our work is to build,
from a model $\mathfrak M$ of homotopy type theory, another model
$\mathfrak M'$ satisfying new principles. Of course, $\mathfrak M'$
should be describable {\em inside} $\mathfrak M$. In set theory, it
corresponds to building {\em inner models} (\cite{kunen}).%
%
In type theory, it can be rephrased in terms of left-exact modalities:
it consists of an operator $\modal$ on types such that for any type
$A$, $\modal A$ satisfies a desired property. If the operator has a
``good'' behaviour, then it is a modality, and the universe of all
types satisfying the chosen property forms a new model of homotopy
type theory.

\section{Modalities}
\label{sec:modalities}

\begin{defi}
  \label{def:modality}
  A left exact modality is the data of
  \begin{enumerate}[(i)]
  \item A predicate $P:\Type \to \HProp$
  \item For every type $A$, a type
    $\modal A$ such that $P(\modal A)$
  \item For every type $A$, a map $\eta_A:A \to
    \modal A$
  \end{enumerate}
  such that
  \begin{enumerate}[(i)]
    \setcounter{enumi}{3}
  \item For every types $A$ and $B$, if $P(B)$ then
    \[ \left\{
        \begin{array}{rcl}
          (\modal A \to B) & \to & (A \to B) \\
          f & \mapsto & f \circ \eta_A
        \end{array} \right. \] %
    is an equivalence.
  \item for any $A:\Type$ and $B:A \to \Type$ such that $P(A)$
    and $\prod_{x:A} P(B x)$, then $P\left( \sum_{x:A} B(x)\right)$
  \item for any $A:\Type$ and $x,y:A$, if $\modal A$ is
    contractible, then $\modal (x=y)$ is contractible.
  \end{enumerate}
  Conditions (i) to (iv) define a {\em reflective subuniverse}, (i) to
  (v) a {\em modality}.
\end{defi}

\begin{prop}\label{prop:mod_prop}
  Here are the properties of modalities.
\end{prop}

\section{Examples of modalities}
\label{sec:modalities-examples}

\subsection{The identity modality}
\label{ssec:id_mod}

Let us begin with the most simple modality one can imagine: the one
doing nothing. We can define it by letting $\modal A \defeq A$ for any type
$A$, and $\eta_A \defeq \idmap$. Obviously, the desired computation
rules are satisfied, so that the identity modality is indeed a
left-exact modality.

It might sound useless to consider such a modality, but it can be
precious when looking for properties of modalities: if it does not
hold for the identity modality, it cannot hold for an abstract one.


\subsection{Truncations}
\label{ssec:truncations}

The first class of non-trivial examples might be the {\em truncations}
modalities.

\subsection{Double negation modality}
\label{ssec:notnot}

The double negation modality $\modal A \defeq \lnot\lnot A$ is a
modality. Unfortunately, it appears that every type is collapsed to an
$\HProp$, thus it cannot be used as-is. The main purpose of this
thesis, in particular chapter~\ref{chap:sheaf} is to extend this
modality into a better one.

\section{New type theories}
\label{sec:new-type-theories}

\begin{prop}\label{prop:consistent}
  
\end{prop}

\section{Truncated modalities}
\label{sec:trunc_modalities}

\section{Translation}
\label{sec:translation}




\chapter{Colimits}
\label{chap:colim}

\epigraph{A comathematician is a device turning cotheorems into
  ffee.}{Mathematical folklore}

As seen in chapter~\ref{chap:hott}, adding sigma-types to type theory
results in adding limits over graphs in the underlying category, and
adding higher inductives typves results in adding colimits over
graphs. If limits has been extensively studied in~\cite{lumsdaine},
theory of colimits was not completely treated.

The following is conjoint work with Simon Boulier and Nicolas
Tabareau, helped by precious discussions with Egbert Rijke.
The sections~\ref{sec:colim} and~\ref{sec:floris} are extended version
of the blog post~\cite{boulier}.

\section{Colimits over graphs}
\label{sec:colim}

As colimits are just dual to limits, it seems that it would be very
easy to translate the work on limits to colimits. Althought, even if
it might be because we are more habituated to manipulate sigma-types
than higher inductive types, it seems way harder.

\subsection{Definitions}
\label{ssec:colim:defi}

Let's recall the definitions of graphs and diagrams over graphs,
introduced in~\cite{lumsdaine}.

\begin{defi}[Graph]\label{defi:graph}
  A {\em graph} $G$ is the data of
  \begin{itemize}
  \item a type $G_0$ of vertices ;
  \item for any $i,j:G_0$, a type $G_1(i,j)$ of edges.
  \end{itemize}
\end{defi}

\begin{defi}[Diagram]\label{defi:diagram}
  A {\em diagram} $D$ over a graph $G$ is the data of
  \begin{itemize}
  \item for any $i:G_0$, a type $D_0(i)$ ;
  \item for any $i,j:G_0$ and all $\phi : G_1(i,j)$, a map $D_1(\phi)
    : D_0(i) \to D_0(j)$
  \end{itemize}
\end{defi}

When the context is clear, $G_0$ will be simply denoted $G$,
$G_1(i,j)$ will be noted $G(i,j)$, $D_0(i)$
will be noted $D(i)$ or $D_i$, and $D_1(\phi)$ will be noted
$D_{i,j}(\phi)$ or simply $D(\phi)$ ($i$ and $j$ can be inferred from
$\phi$).

\begin{exms}
  \item One can consider the following graph, namely the graph of
    (co)equalizers
    \[ \xymatrix{\bullet \ar@<-.5ex>[r] \ar@<.5ex>[r] & \bullet} \]
    Here, $G_0 = \two$, $G_1(\True,\False) = \two$ and other
    $G_1(i,j)$ are empty.

    A diagram over this graph consists of two types $A$ and $B$, and
    two maps $f,g:A \to B$, producing the diagram
    \[ \xymatrix{A \ar@<-.5ex>[r]_g \ar@<.5ex>[r]^f & B} \]
  \item The graph of the mapping telescope is 
    \[ \xymatrix{ \bullet \ar[r] &\bullet \ar[r] &\cdots} \]
    In other words, $G_0=\N$ and $G_1(i,i+1) = \one$.

    A diagram over the mapping telescope is a sequence of types $P:\N
    \to \Type$ together with arrows $f_n:P_n \to P_{n+1}$:
    \[ \xymatrix{ P_0 \ar[r]^{f_0} & P_1 \ar[r]^{f_1} & \cdots} \]
\end{exms}

What we would like now would be to define the colimits of these
diagrams over graphs, that would satisfy type theoretic versions of
usual properties: it should make the diagram commute, and be universal
with respect to this property.
From now on, let $G$ be a graph and $D$ a diagram over this graph.

The commutation of the diagram is easy: the colimit should be the tip
of a cocone.

\begin{defi}[Cocone]\label{defi:cocone}
  Let $Q$ be a type. A cocone over $D$ intro $Q$ is the data of arrows
  $q_i:D_i \to Q$, and for any $i,j:G$ and $g:G(i,j)$, an homotopy 
  $q_j \circ D(g) \homot q_i$.
\end{defi}

If $Q$ and $Q'$ are type with an arrow $f:Q\to Q'$, and if $C$ is a
cocone over $D$ into $Q$, one can easily build a cocone on $D$ into
$Q'$ by postcomposing all maps of the cocone by $f$, giving a map
\newcommand{\postcomposecocone}{\mathrm{postcompose}_{\mathrm{cocone}}}
\[\postcomposecocone : \cocone_D(Q) \to (Q':\Type) \to (Q \to Q') \to \cocone_D(Q') \]
The other way around (from a cocone into $Q'$, give a map $Q\to Q'$)
is exactly the second condition we seek:

\begin{defi}[Universality of a cocone]
  Let $Q$ be a type, and $C$ be a cocone over $D$ into $Q$. $C$ is
  said universal if for any type $Q'$, $\postcomposecocone(C,Q')$ is
  an equivalence.
\end{defi}

We can finally define what it means for $Q$ to be a colimit of $D$.

\begin{defi}[Colimit]\label{defi:colimit}
  A type $Q$ is said to be a colimit of $D$ if there is a cocone $C$
  over $D$ into $Q$, which is universal.
\end{defi}

\begin{exm}
  Let $A$, $B$ be types and $f,g:A\to B$.
  Let $Q$ be the HIT generated by
  $\left|
  \begin{array}{lll}
    q & : & B \to Q \\
    \alpha & : & q \circ f \homot q \circ g
  \end{array} \right. .$
Then $Q$ is a colimit of the coequalizer diagram associated to
$A,B,f,g$. We say that $Q$ is a coequalizer of $f$ and $g$.
\end{exm}

Note that for any diagram $D$, one can build a free colimit of $D$,
namely the higher inductive type $\colim(D)$ generated by
\[ 
  \left|
    \begin{array}{lll}
      \colim & : & \prodD i G {D_i \to \colim(D)} \\
      \alpha_\colim & : & \prodD {i j} G {\prodD g {G(i,j)} {\prodD x
                          {D_i} {\colim_j \circ D(g) \homot \colim_i}}}
    \end{array} \right.
\]


\todo[inline]{Finish colimits}
\subsection{Properties of colimits}
\label{ssec:prop_colim}

\subsection{Truncated colimits}
\label{ssec:trunc_colim}

As said in\todo[fancyline]{Insert here a relevent ref}, we now give a
truncated version of colimits. 
Colimits actually behave well with respect to truncations. Indeed, if $D$ is a
diagram and $P$ a colimit of $D$, then $\|P\|_n$ is the $n$-colimit
of the $n$-truncated diagram $\|D\|_n$. Let's make it more precise.

\begin{defi}[Truncation of a diagram]
  Let $D$ be a diagram over a graph $G$, and $n$ a truncation index.
  Then the diagram $\|D\|_n$ is the diagram over $G$ defined by
  \begin{itemize}
  \item $(\|D\|_n)_0(i) \defeq \|D_0(i)\|_n:\Type_n$
  \item $(\|D\|_n)_1(\phi) \defeq |D_1(\phi)|_n : \|D_0(i)\|_n \to \|D_0(j)\|_n$
  \end{itemize}
\end{defi}

\begin{defi}
  Let $D$ be a diagram over a graph $G$, $P$ be a type, and $C$ a
  cocone over $D$ into $P$. $C$ is said $n$-universal if for any
  $Q:\Type_n$, $\postcompose_{\mathrm{cocone}}(C,Q)$ is an
  equivalence.

  Then, $P$ is said to be a $n$-colimit of $D$ if there is a cocone
  $C$ over $D$ into $Q$ which is $n$-universal.
\end{defi}

We can now give the fundamental proposition linking colimit and
$n$-colimit.

\begin{prop}
  Let $D$ be a diagram, and $P:\Type$.
  Then, if $P$ is a colimit of $D$, $\|P\|_n$ is a $n$-colimit of $\|D\|_n$.
\end{prop}

The proof of this is really straightforward: a cocone over $D$ into
$P$ can be changed equivalently into a cocone over $\|D\|_n$ into $\|P\|_n$, using the
elimination principle~\ref{lem:trunc_elim} of truncations, and then
we can show that the following diagram commutes for any $X:\Type_n$
\[
  \xymatrix{
    \|P\|_n \to X \ar[r] \ar[d]^*[@]{\hbox to 0pt{\hss$\sim$\hss}} & \mathrm{cocone}(\|D\|_n,X) \\
    P \to X \ar[r]^\sim& \mathrm{cocone}(D,X) \ar[u]^*[@]{\hbox to 0pt{\hss$\sim$\hss}}
  }
\]

\begin{rmq}
This result does not hold for limits. If it were
true, then applying it to the following equalizer diagram
\[ \xymatrix{ A \ar@<-.5ex>[r]_{\lambda \_,\,y} \ar@<.5ex>[r]^f & B }\]
with $A,B:\Type$, $f:A\to B$ and $y:B$ would lead to an equivalence
\[ \left\| \sumD a A {f a = y} \right\|_{n} \simeq \sumD a {\|A\|_{n}}
  {|f|_{n}\, a = |y|_{n}}, \]
proving left-exactness of $n$-truncation.  
\end{rmq}

\subsection{Towards highly coherent colimits}
\label{ssec:high_colimit}

\section{Van Doorn's and Boulier's constructions}
\label{sec:floris}

In topos theory, there is a result that we would want to use in
chapter~\ref{chap:sheaf}:
\begin{lem}[{\cite[IV.7.8]{maclanemoerdijk}}]
  In a topos $\mathcal E$, if $f\in\Hom{\mathcal E}(A,B)$ is an epimorphism, then the colimit
  of
  \[ \xymatrix{ A\times_B A \ar@<-.5ex>[r]_-{\pi_2}
      \ar@<.5ex>[r]^-{\pi_1} & A }\]
  is $B$. The pullback $A\times_B A$ is called the kernel pair of $f$.
  
\end{lem}
Unfortunately, this result fails in higher topos ; the kernel pair
should be replaced by the \v{C}ech nerve of $f$.
\[
  \xymatrix{
    \dots \ar@<-1ex>[r]\ar@<-.33ex>[r]\ar@<.33ex>[r]\ar@<1ex>[r] & A\times_B A\times_B A \ar@<-.7ex>[r] \ar[r] \ar@<.7ex>[r]
    & A\times_B A \ar@<-.5ex>[r] \ar@<.5ex>[r] & A \ar[r]^f_{\colim}& B
  }
\]
The issue we face in homotopy type theory is that the definition of
the \v{C}ech nerve, and in general of simplicial objects is a hard
open problem. It involves an infinite tower of coherences, and we
do not know how to handle this. However, there is a way to define a
diagram depending on a map $f$, which colimit is $\im(f)$.

The starting point of the construction is Floris Van Doorn's
construction of proposition truncation~\cite{floris}. 

\begin{prop}[Van Doorn's construction]\label{prop:floris}
  Let $A:\Type$. We define the higher inductive type $TA$ as the
  coequalizer of
  \[ \xymatrix{ A\times A \ar@<-.5ex>[r]_-{\pi_2}
      \ar@<.5ex>[r]^-{\pi_1} & A }.\]
  The colimit of the diagram
  \[ \xymatrix{
      A \ar[r]^-q& TA \ar[r]^-q& TTA \ar[r]^-q& TTTA \ar[r]^-q& \dots
    }\]
  is $\|A\|_{-1}$.
\end{prop}
Let's compare the direct definitions of $\|A\|_{-1}$ and $TA$
\[
  \|A\|_{-1} \left|
    \begin{array}{lll}
      \tr &:& A\to\|A\|_{-1} \\
      \alpha_{\tr} &:& \prodD {x,y}{\|A\|_{-1}}{x=y}
    \end{array}
  \right.
  \qquad
  TA \left|
    \begin{array}{lll}
      q &:& A\to TA \\
      \alpha &:& \prodD {x,y}{A}{q\, x=q\, y}
    \end{array}
  \right.
\]
The definitions are almost the same, except that the path constructor
of $TA$ quantifies over $A$, while the one of $\|A\|_{-1}$ quantifies
over $\|A\|_{-1}$ itself: such a higher inductive type is a {\em
  recursive} inductive type. Thus, proposition~\ref{prop:floris}
allows us to build the truncation in a non-recursive way. The
counterpart is that we have to iterate the
construction. In~\cite{boulier}, we found a way to generalise a bit
this result. The main idea is that iterating the kernel pair
construction will result in a diagram of colimit $\im(f)$.

Now, let $A,B:\Type$ and $f:A\to B$. We define the kernel pair of $f$
as the coequalizer of \[ \xymatrix{ A\times_B A \ar@<-.5ex>[r]_-{\pi_2}
      \ar@<.5ex>[r]^-{\pi_1} & A }.\]
In other words, $\KP(f)$ is the higher inductive type generated by
\[\left|
    \begin{array}{lll}
      \kp &:& A\to \KP(f) \\
      \alpha &:& \displaystyle{\prodD {x,y}{A}{f\, x = f\, y \to kp\,x=\kp\,y}}
    \end{array}
  \right.\]
Using the eliminator of coequalizers, one can build a map $\widehat
f:\KP(f) \to B$, such that the following commutes
\[\xymatrix{
    A \ar[r]^-{\kp} \ar[rd]_-f & \KP(f) \ar[d]^-{\widehat f} \\
    &B
}\]
Then we can build $KP(\widehat f)$ and build a map $\widehat{\widehat
  f}:\KP(\widehat f) \to B$, \etc{}
We have the following result
\begin{prop}[Boulier's construction]\label{prop:cech'}
  For any $f:A\to B$, $\im(f)$ is the colimit of the iterated kernel
  pair diagram of $f$
\[\xymatrix{
  A \ar[r] & \KP(f) \ar[r] & \KP\left(\widehat f\right) \ar[r]& \KP\left(\widehat{\widehat f}\right) \ar[r]& \cdots
}\]
In particular, if $f$ is a surjection, the colimit of this diagram is $B$.
\end{prop}
\begin{proof}
  The main idea of the proof is the equivalence between the diagrams
\[
\xymatrix{
  A \ar[r] & \KP(f) \ar[r] & \KP\left(\widehat f\right) \ar[r]&
  \cdots & \\
  \displaystyle{\sumD y B {\fib f y}} \ar[r] & \displaystyle{\sumD y B {T(\fib f y)}} \ar[r]& \displaystyle{\sumD
  y B {TT(\fib f y)}} \ar[r]&  \cdots &
}
\]

Let's begin by showing the first non-trivial equivalence: 
\begin{equation}
  \label{eq:KP}
  s: \KP(f) \simeq \sumD y B {T(\fib f y)}.  
\end{equation}

$\KP(f)$ is the colimit of $\xymatrix{ A\times_B A \ar@<-.5ex>[r]_-{\pi_2}
  \ar@<.5ex>[r]^-{\pi_1} & A }$, and $\sumD y B {T(\fib f y)}$ is
the colimit of $\xymatrix{ \sumD y B {\fib f y \times \fib f y} \ar@<-.5ex>[r]
  \ar@<.5ex>[r] & \sumD y B {\fib f y} }$.
As the two diagrams are equivalent, their colimits are equivalent. We
will need the following fact, easily checked
\[ \pi_1 \circ s = \widehat f.\]

Now, let's prove the other equivalences. We need the following lemma
\begin{lem}
  Let $X,Y:\Type$ and $\varphi:X\to Y$. Then $\fib{\widehat\varphi}y
  \simeq T(\fib\varphi y)$.
\end{lem}
\begin{prooflem}
  We have the following sequence of equivalences:
  \begin{align*}
    \fib{\widehat\varphi}y 
    &\defeq \sumD x {\KP(\varphi)}{\widehat \varphi\, x = y} \\
    &\simeq \sumD x {\sumD y B {T(\fib \varphi y)}} {\widehat \varphi \circ s^{-1}
      (x) = y} \quad\text{by \ref{eq:KP}} \\
    &\simeq\sumD x {\sumD y B {T(\fib \varphi y)}} {\pi_1(x) = y} \\
    &\simeq T(\fib \varphi y)
  \end{align*}
\end{prooflem}
Then, using the sum-of-fibers property, we can change the iterated
kernel pair of $f$ into
\[
\xymatrix{
\displaystyle{\sumD y B {\fib f y}} \ar[r] & \displaystyle{\sumD y B
  {\fib{\widehat f}y}} \ar[r]& \displaystyle{\sumD
  y B {\fib{\widehat{\widehat f}} y}} \ar[r]&  \cdots &
}
\]
With the just proved lemma, and a bit of induction, we can prove the
desired equivalence of diagrams.

As colimits are stable under dependent sum, we know that the colimit
of the diagram is thus $\sumD y B {Q\, y}$, where $Q$ is the colimit
of
\[\xymatrix{\displaystyle{{\fib f y}} \ar[r] & \displaystyle{{T(\fib f
        y)}} \ar[r]& \displaystyle{{TT(\fib f y)}} \ar[r]&  \cdots
    &}\]
But proposition~\ref{prop:floris} asserts that $Q \simeq \|\fib f
y\|_{-1}$, and the result is proved.
\end{proof}

The main issue with Van Doorn's construction is
that is does not preserve truncations levels at all. For example, when
computing $\|\one\|_{-1}$, the first step is $T\one \simeq \Sone$,
which is a $1$-type. 
Asking for preservation of all truncation levels along the diagram
might be too much, but the least we could ask is that when starting
with a $P:\HProp$, the diagram should be the constant diagram $P\to
P \to P\to \cdots$. The Boulier counterpart of this is, when starting
with an embedding $f$, all $\widehat f$ are embeddings. 

This can be achieve by changing operators $T$ and $\KP$, by asking
them to preserve identities:
\[
  TA~\left|
    \begin{array}{lll}
      q&:& A\to TA \\
      \alpha&:& \displaystyle{\prodD {x,y} A {q\, x = q\, y}} \\
      \alpha_1&:& \displaystyle{\prodD x A {\alpha(x,x) = 1}}
    \end{array}
  \right.
  \quad ; \quad
  \KP(f)~\left|
    \begin{array}{lll}
      \kp&:&A\to \KP(f) \\
      \alpha&:& \displaystyle{\prodD {x,y} A {f\, x = f\, y \to q\, x = q\, y}} \\
      \alpha_1&:& \displaystyle{\prodD x A {\alpha(x,x,1) = 1}}
    \end{array}    
  \right.
\]
Then, the following is still true

\begin{prop}[Boulier's construction]\label{prop:cech}
  For any $f:A\to B$, $\im(f)$ is the colimit of the iterated kernel
  pair diagram of $f$
\[\xymatrix{
  A \ar[r] & \KP(f) \ar[r] & \KP\left(\widehat f\right) \ar[r]& \KP\left(\widehat{\widehat f}\right) \ar[r]& \cdots
}\]
Moreover, if $f$ is an embedding, $\widehat f$ also is.
\end{prop}
The proof is almost the same as for proposition~\ref{prop:cech'},
except that the new constructors in the higher inductives types
introduce another level of coherence, which is very technical to handle.

\section{Towards groupoid objects}
\label{sec:groupoid}
\chapter{Sheaves in homotopy type theory}
\label{chap:sheaf}

\section{Forcing in type theory}
\label{sec:forcing}

\section{Sheaves in topoi}
\label{sec:sheaf_topos}

\section{Sheaf theory}
\label{sec:sheaf_hott}
\chapter{Conclusion and future works}
\label{chap:conclusion}

Let us begin this conclusion by a summary of this thesis. Homotopy
type theory is a new research domain, and consists of Martin-Löf type
theory where we give a homotopical interpretation of identity types,
together with the univalence axiom, linking equivalence of types with
their equality, and higher inductive types, giving us a way to build
non-trivial equalities. 
It seems that there exists a very strong link between higher topos
theory~\cite{lurie} and this theory. More precisely, homotopy type
theory is expected to be the internal language of
$(\infty,1)$-toposes.

In homotopy type theory, types are classified
by their {\em truncation level}, representing the complexity of its
iterated loop spaces. In particular, if $X$ is $(n+1)$-truncated, then
each $x = y$ with $x,y:X$ is $n$-truncated. We will use this property
to build an operator on all truncated types by induction on the
truncation level.

The operator we want to build is a {\em modality}. Modalities are
generalized version of localizations, which themselves are a way to
characterize equivalently the notion of subtopos; this equivalence
still holds in higher topos theory~\cite[Section 6.2.2]{lurie}. In
homotopy type theory, a modality is an operator $\modal$ on $\Type$,
together with unit maps $\eta : \prodD X \Type {X \to \modal X}$
satisfying properties. They can be seen just as idempotent monads.
Relying on the equivalence between type theory and
$(\infty,1)$-toposes, one can conjecture that modalities -- actually,
accessible left-exact modalities -- induces reflective sub-type
theories. In this thesis, we want to describe a classical type theory
(\ie{} with the law of excluded middle) a sub-type theory of homotopy
type theory, using a modality.
We already know that it is possible in topos theory: take any topos
$\mathcal T$, and the double-negation Lawvere-Tierney topology, and
build the topos of sheaves $\Sh{\lnot\lnot}(\mathcal T)$. Then the
latter is a boolean topos (\ie{} satisfying the LEM).

The main idea of our work is to notice that Lawvere-Tierney topologies
on a topos, which are operator on the subobject classifier, can be
seen, in the setting of homotopy type theory, as modalities restricted
to $\HProp$, the second layer of the stratification of
types. Moreover, the sheafification functor corresponds to extend this
truncated modality to $\HSet$, the next layer. Thus, we believed that
it was possible to extend it again to the next level, \etc{} to
finally give a modality on all truncated types. Actually, {\em modulo}
some changes in several proofs -- in particular the proof involving
kernel pairs of arrows -- and some sophistications in proofs, the
method described in the topos theoretic setting can be repeated
infinitely to build a truncated modality on all level of our
stratification.

Unfortunately, the actual management of universes by Coq does not
allow us to formalize this result completely, but most of the proof is
computer-checked.
Some parts can only be checked using the (inconsistent)
\code{type-in-type} option, allowing $\Type^i : \Type^i$, but the
whole inductive definition cannot be checked at all. However, the key
points of the definition of the functor are formalized.

Nevertheless, our work still need enhancements:
\paragraph*{Extension to Type.}
At the moment, our sheafification functor only handles truncated type,
and we have to compose it with truncations. It would be way more
satisfying to be able to define it on whole $\Type$ left-exactly. The
main issue is that some types, which are not $n$-truncated for any
$n$, are not even the limit of their
truncations~\cite{morelvv}. Therefore, there seems to be no way to
create a link between a non-truncated type and truncated types, to
extend our inductive definition.
It might be possible to have such a link using axioms such as
Whitehead's principle~\cite[Section 8.8]{hottbook} or Postnikov
principle~\cite[Section 5.5.6]{lurie}, and use it to build a real
modality on $\Type$.

\paragraph*{Lawvere-Tierney sheaves in higher topos theory.}
If we rely on the leitmotiv
\begin{quote}
  Homotopy type theory is the internal language of $(\infty,1)$-topos,
\end{quote}
we could transpose our work to higher topos theory. As there are more
tools in topos theory (\eg{} we can access the definitional equality),
it could be a first step in solving the previous future work.
This kind of ``reverse engineered'' proof has already been done for a
proof of the Blakers-Massey theorem by Charles Rezk~\cite{rezk-BM},
inspired by the homotopy-type-theoretic proof by Peter LeFanu
Lumsdaine, Eric Finster and Dan Licata.

\paragraph*{Lawvere-Tierney subsumes Grothendieck?}
In topos theory, there are two different notions of sheaves: the
Grothendieck sheaves and the Lawvere-Tierney sheaves.
The former is a topological, geometrical concept, while the latter is
rather a logical concept.
Grothendieck sheaves are based on {\em Grothendieck
  topologies}~\cite[Chapter III]{maclanemoerdijk}, and one can show
that Lawvere-Tierney topologies on a presheaf topos
$\mathbf{Sets}^{\mathbf{C}^{\mathrm{op}}}$ correspond exactly to
Grothendieck topologies on $\mathbf{C}$. Then, we have the following:
\begin{thm}[{\cite[{Section V.4, theorem 2}]{maclanemoerdijk}}]
  \label{thm:subsume}
  Let $\mathbf{C}$ is a small category and $j$ a Lawvere-Tierney
  topology on $\mathbf{Sets}^{\mathbf{C}^{\mathrm{op}}}$, while $J$ is
  the corresponding Grothendieck topology on $\mathbf{C}$. Then a
  presheaf $P$ is a sheaf for $j$ iff $P$ is a $J$-sheaf.
\end{thm}
The concept of Grothendieck sheaf and Grothendieck sheafification
already exists in $(\infty,1)$-topos~\cite[Section 6.2.2]{lurie}. 
It would be nice to check if theorem~\ref{thm:subsume} still holds,
either in the setting of homotopy type theory or in
the setting of higher topos. The former requires to formalize
Grothendieck topologies, sheaves and sheafification from higher topos
theory to homotopy type theory, while the latter requires to work on
the previous point.





%%% Local Variables:
%%% mode: latex
%%% TeX-master: "main"
%%% End:


\backmatter%
% \listoftodos[What remains to be done]
\printnomenclature[10em]
\printbibliography


\end{document}

%%% Local Variables:
%%% mode: latex
%%% TeX-master: t
%%% End:
